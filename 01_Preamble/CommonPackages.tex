%%% File encoding is UTF8
%%% You can use special characters just like ä,ü and ñ

% Input encoding is 'latin1' (Latin 1 - also known as ISO-8859-1)
% CTAN: http://www.ctan.org/pkg/inputenc
% 
% A newer package is available - you may look into:
% \usepackage[x-iso-8859-1]{inputenc}
% CTAN: http://www.ctan.org/pkg/inputenx
%\usepackage[latin1]{inputenc}
\usepackage[utf8]{inputenc}

% Font Encoding is 'T1' -- important for special characters such as Umlaute ü or ä and special characters like ñ (enje)
% CTAN: http://www.ctan.org/pkg/fontenc
\usepackage[T1]{fontenc}

% Language support for 'english' (alternative 'ngerman' or 'french' for example)
% This case is in spanish
% CTAN: http://www.ctan.org/pkg/babel
%\usepackage[english]{babel} 
\usepackage[spanish
          , es-tabla
          , es-noindentfirst]{babel} 

% Doing calculations with LaTeX units -- needed for the vertical line in the footer
% CTAN: http://www.ctan.org/pkg/calc
\usepackage{calc}

% Extended graphics support 
% There is also a package named 'graphics' - watch out!
% CTAN: http://www.ctan.org/pkg/graphicx
\usepackage{graphicx}

% Extendes support for floating objects (tables, figures), adds the [H] placing option (\begin{figure}[H]) which palces it "Here" (without any doubt).
% CTAN: http://www.ctan.org/pkg/float
\usepackage{float}

% Extended color support
% I use the command \definecolor for example. 
% Option 'Table': Load the colortbl package, in order to use the tools for coloring rows, columns, and cells within tables.
% CTAN: http://www.ctan.org/pkg/xcolor
\usepackage[table]{xcolor} 

% Nice tables
% CTAN: http://www.ctan.org/pkg/booktabs
\usepackage{booktabs}

\usepackage{colortbl}

\usepackage{multirow,siunitx}

% Better support for ragged left and right. Provides the commands \RaggedRight and \RaggedLeft. 
% Standard LaTeX commands are \raggedright and \raggedleft
% http://www.ctan.org/pkg/ragged2e
\usepackage{ragged2e}

% Create function plots directly in LaTeX
% CTAN: http://www.ctan.org/pkg/pgfplots
\usepackage{pgfplots}
\pgfplotsset{compat=1.11}

% Main Font used: Arial
\usepackage{uarial}
\usepackage[expert]{mathdesign}
\renewcommand{\familydefault}{\sfdefault}

% tocloft ? Control table of contents, figures, etc
% CTAN: https://ctan.org/pkg/tocloft
%\usepackage{tocloft}

% Configure Microtype
\usepackage[activate={true,nocompatibility},final,tracking=true,kerning=true,spacing=true,factor=1100,stretch=10,shrink=10]{microtype}
% activate={true,nocompatibility} - activate protrusion and expansion
% final - enable microtype; use "draft" to disable
% tracking=true, kerning=true, spacing=true - activate these techniques
% factor=1100 - add 10% to the protrusion amount (default is 1000)
% stretch=10, shrink=10 - reduce stretchability/shrinkability (default is 20/20)

\SetProtrusion{encoding={*},family={bch},series={*},size={6,7}}
              {1={ ,750},2={ ,500},3={ ,500},4={ ,500},5={ ,500},
               6={ ,500},7={ ,600},8={ ,500},9={ ,500},0={ ,500}}

\SetExtraKerning[unit=space]
{encoding={*}, family={bch}, series={*}, size={footnotesize,small,normalsize}}
{\textendash={400,400}, % en-dash, add more space around it
	"28={ ,150}, % left bracket, add space from right
	"29={150, }, % right bracket, add space from left
	\textquotedblleft={ ,150}, % left quotation mark, space from right
	\textquotedblright={150, }} % right quotation mark, space from left

\SetExtraKerning[unit=space]
{encoding={*}, family={qhv}, series={b}, size={large,Large}}
{1={-200,-200}, 
	\textendash={400,400}}

% Use French spacing
\frenchspacing



% https://www.ctan.org/pkg/biblatex-ieee
\usepackage[backend=biber
		  %, style=ieee % Esta bien, pero no agrupa las citaciones
		  %, style=numeric-comp %Revisar, porque algunas referencias no las tira correctas
		  %Antes: IEEE
		  %, bibstyle=ieee  %Combinacion de ambas
		  %, citestyle=numeric-comp
		  %AHORA: APA
		  , style=apa
		  , hyperref=true
		  , url=false
		  , isbn=false
		  , backref=true
		 %, style=custom-numeric-comp
		  , citereset=chapter
		  , maxcitenames=3
		  , maxbibnames=100
		  , block=none
		  %, sorting=none % El orden de las citas en IEEE es según el orden de aparicion en el texto
		  , sortcites=true
		  , sorting=nyt
		  % APA no tiene referencias cruzadas, si se quiere desactivar, comentar apabackref
		  , apabackref=true
		  , language=spanish
]{biblatex}
\bibliography{bibfile}

% Separacion entre entradas de la referencia
\setlength\bibitemsep{\baselineskip}

\DeclareLanguageMapping{spanish}{spanish-apa}

\usepackage[style=english]{csquotes}

%TODO: Cambiar de bibliografia a Referencias Bibliograficas
\renewcommand\bibname{Referencias bibliogr\'aficas}
\DefineBibliographyStrings{spanish}{%
	andothers = {\em et\addabbrvspace al\adddot}
}

\AtEveryBibitem{%
	\ifboolexpr{test {\ifentrytype{article}} and not test {\iffieldundef{doi}}}
	{\clearfield{number}}
	{}%
}

\usepackage{pgfplotstable}

%FIX: dont work
\usepackage[noabbrev,capitalize,nameinlink,spanish]{cleveref}
\crefname{table}{\spanishtablename}{\spanishtablename}

\usepackage{subcaption}

\usepackage{neuralnetwork}

\usepackage[linesnumbered
		   , algoruled
	       , vlined
	       , boxed
	       , algochapter
	       , commentsnumbered
	       , spanish
	       , onelanguage
]{algorithm2e}

\usepackage{pgfgantt}
\ganttset{
	, y unit title=0.5cm
	, y unit chart=0.7cm
	, vgrid,hgrid
	, progress=today
	, group/.append style={orange}
	, milestone/.append style={red}
	, progress label node anchor/.append style={text=red}
	, bar/.style={draw=black, fill=gray!50}
	, incomplete/.style={draw=black, fill=white}
	, title height=1
	, title label font=\bfseries\footnotesize
	, bar/.style={fill=black}
	, bar height=0.7
	, today label=HOY
	, group right shift=0
	, group top shift=0.7
	, group height=.3
	, group peaks width={0.2}
	, inline
} 

\usepackage{graphicx}
\usepackage{xcolor}

\usepackage{amsmath,amssymb}

\usepackage{tabulary}

%Fancy tables
\usepackage{array,booktabs}
\newcolumntype{L}{@{}>{\kern\tabcolsep}l<{\kern\tabcolsep}}
\usepackage{colortbl}
\usepackage{xcolor}


%Tikz packages
\usepackage{tikzpeople}
%\usepackage{forest}
%%%%%%%%%%%%%%%%%%%%%%%%%%%%%%%%%%%%%%%%%%%%

\usetikzlibrary{fit}
\usetikzlibrary{arrows.meta}
\usetikzlibrary{positioning}
\usetikzlibrary{shapes.geometric}
\usetikzlibrary{decorations.pathreplacing}
\usetikzlibrary{calc} 
\usetikzlibrary{patterns}

\tikzstyle{roundBox} = [text width=6em, fill=white, minimum height=5em, rounded corners, text centered, draw=black]

\tikzset{
	arrow/.style={-latex, shorten >=0pt, shorten <=0pt},
	arrowboth/.style={<->/.tip = Latex, shorten >=0pt, shorten <=0pt}
}

\usetikzlibrary{decorations.pathreplacing,positioning,arrows}

% Define some styles
\tikzset{
 block/.style = {
    %rectangle, draw, fill=blue!30,
    rectangle, draw, fill=bgcolor1,
    text width=7em, 
    text centered, 
    rounded corners, 
    minimum height=4em},
 line/.style = {
    draw, -latex'},
 my brace/.style = {
   decorate,decoration={brace,amplitude=10pt},
   shorten >=3pt, shorten <=3pt
 },
 bottom label/.style = {
    black, midway, yshift=-3ex, font=\itshape,
    text width = 3cm, text centered,
 },
 top label/.style = {
    black, midway, above, yshift=3ex, font=\large,
 },
}

%\usepackage{xcolor}

\definecolor{bgcolor0}{RGB}{255, 231, 231} %red!10
\definecolor{strokecolor0}{RGB}{255, 181, 181} %red!30

\definecolor{bgcolor1}{RGB}{231, 231, 255} %blue!10
\definecolor{bgcolor2}{RGB}{255, 243, 231} %orange!10

\definecolor{unired}{RGB}{165,0,0}
\definecolor{bgcolor3}{RGB}{238, 205, 205} %unired!20

\tikzset{
	cmdarrow/.style={-latex, shorten >=0pt, shorten <=0pt, color = unired},
	eventarrow/.style={-latex, shorten >=0pt, shorten <=0pt, color = blue!60},
	queryarrow/.style={-latex, shorten >=0pt, shorten <=0pt, color = blue!60},
	cmdcircle/.style={circle, draw, minimum size=5, inner sep=0pt, fill=unired},
	querycircle/.style={circle, draw, minimum size=5, inner sep=0pt, fill=blue!60},
	eventcircle/.style={circle, draw, minimum size=5, inner sep=0pt, fill=blue!60}
}

\definecolor{cmdlabel}{RGB}{165, 0, 0} 
\definecolor{eventlabel}{RGB}{102, 102, 255}
\definecolor{querylabel}{RGB}{102, 102, 255} 


\usepackage{amssymb}

% Custom settings and commands
% \newcommand{\y}{\raisebox{-0.12cm}{\begin{tikzpicture}\node{\includegraphics[height=0.25cm]{/home/jeroen/phd/latex/thesis//dev/tickmark.pdf}};\end{tikzpicture}}}
\newcommand{\y}{\checkmark}
\newcommand{\n}{}
