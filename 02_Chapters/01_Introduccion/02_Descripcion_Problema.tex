\section{Descripción del problema}
\label{sec:descripcion-problema}
En el contexto de la enseñanza de la alfabetización informacional, las evaluaciones de los cursos se centran principalmente en los resultados de los estudiantes sin tomar en cuenta el proceso formativo y factores asociados que podrían influir directa o indirectamente sobre los resultados finales y el desempeño de búsqueda de los alumnos.  

En el contexto del proyecto de investigación iFuCo, el cual pretende realizar un análisis cuantitativo y cualitativo de la alfabetización informacional y las competencias de búsqueda en línea en estudiantes de enseñanza básica\footnote{En otros países es conocido como enseñanza primaria.} en los países de Chile y Finlandia, surgen las siguientes interrogantes (\ingles{research questions}, RQ desde ahora en adelante):

\begin{description}
	\item [RQ 1] ¿De qué manera se puede estimar durante el proceso de aprendizaje de competencias informacionales la influencia de diversos factores en el desempeño de búsqueda de la información de los estudiantes?
	\item [RQ 2] ¿En qué medida es posible detectar situaciones anormales de conducta, y determinar las causas que llevan a un estudiante a fallar durante el proceso de búsqueda de información? 
	\item [RQ 3] ¿De qué manera se puede implementar un módulo de clasificación y predicción del desempeño de los estudiantes en la búsqueda de información en herramientas de apoyo de la alfabetización informacional para proporcionar una retro evaluación oportuna a estudiantes y docentes?
\end{description}