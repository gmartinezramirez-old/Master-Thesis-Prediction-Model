\section{Objetivos y alcances de la solución}
\label{sec:objetivos}

\subsection{Objetivo general}
\label{subsec:objetivo-general}
Diseñar y evaluar un modelo predictivo del desempeño de búsqueda de información en línea de estudiantes de enseñanza básica.

\subsection{Objetivos específicos}
\label{subsec:objetivo-especificos}

\begin{enumerate}
	\item Realizar una revisión bibliográfica sobre trabajos recientes relacionados con minería de datos en el contexto educacional.
	\item Realizar una exploración, limpieza, pre-procesamiento y transformación de los datos recopilados por la plataforma NEURONE (acrónimo de oNlinE inqUiry expeRimentatiON systEm).
	\item Seleccionar características de comportamiento de búsqueda de los estudiantes para la construcción de modelos predictivos.
	\item Construir modelos para la predicción del desempeño de búsqueda en línea de estudiantes de educación básica.
	\item Evaluar los modelos predictivos del desempeño de búsqueda en linea de estudiantes de educación básica.
	\item Implementar los modelos predictivos en la plataforma NEURONE.
\end{enumerate}

\subsection{Alcances}
\label{subsec:alcances}
Los modelos se construyen a partir de un conjunto de datos específicos, estos datos tienen su propio contexto y origen que limitan la generalización de los modelos a construir. A continuación, se describen las principales limitaciones y alcances de la solución.

\begin{enumerate}
	\item El curso de alfabetización informacional y sus respectivos registros de datos, pertenecen al proyecto iFuCo, el cual es un trabajo colaborativo entre universidades de Finlandia (University of Tampere, University of Jyväskylä y University of Turku) y de Chile (Universidad de Santiago de Chile y Pontificia Universidad Católica de Chile). 
	\item Los registros de datos provienen de un estudio enmarcado en un curso de alfabetización en información, aplicado al área de Ciencia y Ciencias Sociales, en ambos países.
	\item Los datos son recolectados y almacenados por un sistema externo llamado NEURONE (\ingles{oNlinE inqUiry expeRimentatiON systEm}), trabajo de memoria de un estudiante de la carrera de Ingeniería de Ejecución en Computación e Informática de la Universidad de Santiago de Chile.
	\item La solución funciona como un sistema predictor del desempeño del estudiante en la búsqueda de información, sin ofrecer acciones correctivas en caso de bajo desempeño.
\end{enumerate}