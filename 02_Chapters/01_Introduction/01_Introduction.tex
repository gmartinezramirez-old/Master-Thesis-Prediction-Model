%%% File encoding is UTF8

\chapter{Introducción}
\label{ch:introduccion}
% Activate arabic numbering (e. g. 12)	
\pagenumbering{arabic}
% Start with page 1 (I)
\setcounter{page}{1}
\section{Antecedentes y motivación}
\label{sec:antecedentes-motivacion}
La búsqueda exploratoria es un proceso que aparece cuando un usuario desea realizar una búsqueda de información en Internet, pero no tiene conocimientos específicos del área o tiene solamente una idea vaga de lo que quiere, por lo cual hace una consulta tentativa para navegar hacia resultados relevantes.

A medida que el volumen de la información existente en Internet aumenta, las formas de recuperación de información se han mejorado. A pesar de esto, los motores de búsqueda no entregan información de una forma eficaz para responder las necesidades de información de los usuarios actuales.

A medida que el volumen de la información existente en Internet aumenta, las formas de recuperación de información se han mejorado. A pesar de esto, los motores de búsqueda no entregan información de una forma eficaz para responder las necesidades de información de los usuarios actuales.

A medida que el volumen de la información existente en Internet aumenta, las formas de recuperación de información se han mejorado. A pesar de esto, los motores de búsqueda no entregan información de una forma eficaz para responder las necesidades de información de los usuarios actuales.

A medida que el volumen de la información existente en Internet aumenta, las formas de recuperación de información se han mejorado. A pesar de esto, los motores de búsqueda no entregan información de una forma eficaz para responder las necesidades de información de los usuarios actuales.

A medida que el volumen de la información existente en Internet aumenta, las formas de recuperación de información se han mejorado. A pesar de esto, los motores de búsqueda no entregan información de una forma eficaz para responder las necesidades de información de los usuarios actuales.

\section{Descripción del problema}
\label{sec:descripcion-problema}

\section{Solución propuesta}
\label{sec:solucion-propuesta}

\subsection{Características de la solución}
\label{subsec:caracteristicas-solucion}

\subsection{Propósito de la solución}
\label{subsec:proposito-solucion}

\section{Objetivos y alcances de la solución}
\label{sec:objetivos}

\subsection{Objetivo general}
\label{subsec:objetivo-generla}

\subsection{Objetivos específicos}
\label{subsec:objetivo-especificos}
\begin{itemize}
	\item Objetivo especifico 1
	\item Objetivo especifico 2
	\item Objetivo especifico 3
\end{itemize}

\subsection{Alcances}
\label{subsec:alcances}
\begin{enumerate}
	\item Alcance 1
	\item Alcance 2
	\item Alcance 3
\end{enumerate}

\section{Metodología y herramientas utilizadas}
\label{sec:metodologia-herramientas}


\subsection{Metodología a usar}
\label{subsec:metodologia}

\subsection{Herramientas de desarrollo}
\label{subsec:herramientas}

\subsubsection*{\textit{Software}}

\subsubsection*{\textit{Hardware}}

\section{Organización del documento}
\label{sec:organizacion}