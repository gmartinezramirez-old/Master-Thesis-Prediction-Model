\section{Resumen}
\label{sec:marco_estado_arte_resumen}
Las competencias informacionales son un conjunto de habilidades asociadas al descubrimiento  reflexivo de la información. Resulta fundamental comprender cómo la información se produce, evalúa, utiliza y comparte. Existen habilidades asociadas puntualmente a tareas de investigación, basadas en la indagación, desarrolladas en Internet (\ingles{online inquiry}) para encontrar, evaluar críticamente, sintetizar y comunicar la información de una manera correcta.  

A partir de lo anterior, la revisión bibliográfica abarca diferentes enfoques. En primer lugar, se aprecia como la enseñanza de las competencias informacionales ha ido migrando de la educación superior a intervenciones en secundaria y en casos puntuales al sector primario. En segundo lugar, se describen estudios que en base a sistemas recopilan información indirecta de los estudiantes, por ejemplo, consultas realizadas en un motor de búsqueda, teclas presionadas, rastreo ocular, tareas de búsqueda, tiempos de permanencia o de respuestas y uso de sitios \ingles{web}, entre otras. En tercer lugar, es la minería de datos educacional, la cual permite explorar datos provenientes de diversas tecnologías educacionales para entender y evaluar habilidades, aprendizaje y por sobre todo comportamiento de los estudiantes. Finalmente, el cuarto enfoque, profundiza en los estudios realizados para predecir el desempeño de los estudiantes, mediante técnicas y algoritmos de minería de datos.   
%Como marco conceptual se definen conceptos importantes que son utilizados en el estudio. Primero la experiencia de usuario dado que implica emociones y actitudes de una persona con respecto al producto o servicio que se esté utilizando suele medirse con instrumentos que requieren una participación directa del usuario como encuestas. Con respecto al rendimiento en el área de recuperación de información se utilizan métricas para evaluarlo a partir fórmulas que usan la cantidad de documentos clasificados y documentos correctamente clasificados.

%Las competencias informacionales son un conjunto de habilidades asociadas al descubrimiento reflexivo de la información. Resulta fundamental comprender cómo la información se produce, evalúa, utiliza y comparte. Existen habilidades asociadas puntualmente a tareas de investigación, basadas en la indagación, desarrolladas en Internet (\ingles{online inquiry}) para encontrar, evaluar críticamente, sintetizar y comunicar la información de una manera correcta.

%En el estado del arte se realiza una revisión de distintos estudios de usuario en los que se trabaja con representaciones alternativas de documentos, donde se observaron distintos resultados. Por ejemplo, Nguyen y Zhang (2006) encontraron que, al utilizar una representación visual de documentos, la clasificación de documentos relevantes mejoraba o Tilsner (2009) quien observó un rendimiento mayor con menores tiempos de respuesta utilizando una interfaz visual. Por otro lado, Sebrechts, Cugini, y Laskowski (1999) obtuvieron un rendimiento similar contrastando una interfaz visual con una tradicional. Por otro lado, se ve que en el mercado ya existen algunas herramientas de visualización como Zakta o oSkope las que pueden resultar beneficiosas si se usan para buscar ciertos tipos de información.

Finalmente, en base a la motivación que guía este estudio, se plantean ciertas preguntas que fundamentan la realización de este trabajo: ¿De qué manera se puede estimar durante el proceso de aprendizaje de competencias informacionales la influencia de diversos factores en el desempeño de búsqueda de la información de los estudiantes?, ¿En qué medida es posible detectar situaciones anormales de conducta, y determinar las causas que llevan a un estudiante a fallar durante el proceso de búsqueda de información?, y ¿De qué manera se puede implementar un módulo de clasificación y predicción del desempeño de los estudiantes en la búsqueda de información en herramientas de apoyo de la alfabetización informacional para proporcionar una retro evaluación oportuna a estudiantes y docentes?.

