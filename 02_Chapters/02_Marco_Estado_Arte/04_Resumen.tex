\section{Resumen}
\label{sec:marco_estado_arte_resumen}
En el presente capítulo

Como marco conceptual se definen conceptos importantes que son utilizados en el estudio. Primero la experiencia de usuario dado que implica emociones y actitudes de una persona con respecto al producto o servicio que se esté utilizando suele medirse con instrumentos que requieren una participación directa del usuario como encuestas. Con respecto al rendimiento en el área de recuperación de información se utilizan métricas para evaluarlo a partir fórmulas que usan la cantidad de documentos clasificados y documentos correctamente clasificados.

Las competencias informacionales son un conjunto de habilidades asociadas al descubrimiento reflexivo de la información. Resulta fundamental comprender cómo la información se produce, evalúa, utiliza y comparte. Existen habilidades asociadas puntualmente a tareas de investigación, basadas en la indagación, desarrolladas en Internet (\ingles{online inquiry}) para encontrar, evaluar críticamente, sintetizar y comunicar la información de una manera correcta.

En el estado del arte se realiza una revisión de distintos estudios de usuario en los que se trabaja con representaciones alternativas de documentos, donde se observaron distintos resultados. Por ejemplo, Nguyen y Zhang (2006) encontraron que, al utilizar una representación visual de documentos, la clasificación de documentos relevantes mejoraba o Tilsner (2009) quien observó un rendimiento mayor con menores tiempos de respuesta utilizando una interfaz visual. Por otro lado, Sebrechts, Cugini, y Laskowski (1999) obtuvieron un rendimiento similar contrastando una interfaz visual con una tradicional. Por otro lado, se ve que en el mercado ya existen algunas herramientas de visualización como Zakta o oSkope las que pueden resultar beneficiosas si se usan para buscar ciertos tipos de información.

Finalmente, en el marco de investigación se plantean preguntas de investigación relacionadas con cómo las personas perciben su interacción con resultados de búsqueda de información a través de representaciones visuales y si es posible mejorar aspectos como la experiencia de usuario el rendimiento utilizando este tipo de representación. Lo anterior lleva a las dos hipótesis con las que se trabaja en este proyecto que de forma resumida son:

