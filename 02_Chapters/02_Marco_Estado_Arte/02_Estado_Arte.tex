\section{Estado del arte}
\label{sec:estado_arte}
El propósito de esta sección es presentar los últimos trabajos realizados en la línea de investigación que se ha planteado en la sección y capítulo anterior. La búsqueda de estos trabajos considera: la exploración de trabajos con estudios que midan experiencia de usuario y/o medidas de rendimiento en la utilización de interfaces no-tradicionales operadas con el cuerpo para la realización de actividades en distintos contextos.

\subsection{Alfabetización informacional}
La enseñanza de competencias informacionales o alfabetización informacional, se imparte  principalmente por bibliotecas universitarias, y en un menor grado en la etapa escolar obligatoria. A pesar de esto, algunas universidades suelen considerar las competencias informacionales como un requisito de entrada, aprendidas antes de iniciar los estudios universitarios, variando su importancia según los principios de cada carrera. Un ejemplo de esto, es la investigación de Weiner (2014) en la cual se utilizaron encuestas para determinar las competencias informacionales de interés en los programas de colleges y universitarios. En el trabajo de Smith  et al. (2013) queda de manifiesto que la alfabetización informacional en enseñanza escolar no necesariamente está vinculada con las reales necesidades universitarias. En dicho trabajo, un grupo de estudiantes canadienses de último año, pertenecientes a escuelas que desarrollan las competencias informacionales, fueron sometidos a un test universitario que mide estas competencias. Los resultados evidencian que los alumnos no se están graduando con el adecuado nivel de alfabetización informacional.  

A nivel nacional, la enseñanza de estas competencias se realiza principalmente por parte de las bibliotecas universitarias, por lo cual, Marzal y Sauriana (2015) realizan un análisis de la situación actual de los cursos y políticas de alfabetización informacional de las bibliotecas pertenecientes  al Consejo Nacional de Educación. Se concluye que no existe uniformidad en las definiciones y marcos teóricos utilizados. Por otro lado, los estudiantes universitarios chilenos presentan dificultades con las competencias informacionales (p. ej. aprenderlas, pero no utilizarlas), hecho que se vio reflejado por el estudio interno realizado por la Universidad de Playa Ancha (Urra & Castro, 2016) sobre el impacto de la enseñanza de la alfabetización informacional en dicha universidad. Una de las posibles causas de por qué los estudiantes tienen dificultades es el hecho de que en los colegios no se prioriza la generación de conocimiento, sino la reiteración de la información.   

En Finlandia, la enseñanza de alfabetización informacional también inicia en las bibliotecas académicas. Durante los años 1990 a 2002 tuvo efecto la transición de educar en el uso de los sistemas de bibliotecas a la enseñanza de competencias informacionales (Sinikara & Järveläinen, 2003). A razón de esto, los primeros estudios y cambios curriculares se enfocaron en este nivel. Por ejemplo, el proyecto “Standardizing the management of the information literacy 2001–2003” de la biblioteca de la Universidad de Helsinki realizó una traducción al finlandés de los estándares de alfabetización internacional propuestos por la ARCL (Asociación de Bibliotecas Universitarias y de Investigación) mientras que en 2004 las bibliotecas universitarias lanzaron un proyecto nacional para la creación de un currículum universitario de competencias informacionales (Kämäräinen & Saarti, 2013).

El siguiente paso en Finlandia fue enfocar los estudios y proyectos al nivel de secundaria. El trabajo realizado por Sormunen, Alamettälä, y Heinström (2013) es ejemplo de la incorporación de las competencias informacionales de forma transversal al proceso educativo, bajo el enfoque de enriquecer el aprendizaje al momento de realizar tareas grupales en el aula, aplicado a la escritura de artículos en clases de historia y literatura. Por otro lado, Alamettälä (2015) propone un plan de trabajo para la enseñanza formal de alfabetización informacional en los primeros niveles de la educación secundaria finlandesa.  

Actualmente, a nivel educacional primario, tres universidades finlandesas (University of Tampere, University of Jyväskylä y University of Turku) en conjunto con dos universidades chilenas (Universidad de Santiago de Chile y Pontificia Universidad Católica) están realizando una intervención en el aula respecto las competencias informacionales en quinto y sexto año básico (CONICYT, 2015), para establecer un marco de enseñanza de estas competencias, que pueda derivar en propuestas públicas y cambios curriculares. Puntualmente el trabajo realizado por Sormunen, González-Ibáñez y Kiili (2017) introduce nuevas formas de evaluar las competencias de investigación en línea (online inquiry) específicamente en búsqueda, evaluación, recolección y síntesis de la información.

\subsection{Comportamiento de búsqueda de información de estudiantes}
Con la incorporación de herramientas digitales en la enseñanza escolar fue necesario evaluar su aporte en el aprendizaje. En general se analizan las mediciones respecto a los datos proporcionados por los estudiantes, ya sea de forma directa (cuestionarios) o indirecta (datos generados al utilizar un sistema computacional).

Respecto de las mediciones de los datos generados por los estudiantes, Henrie, Halverson y Graham (2015) las clasifican en tres categorías: comportamiento, cognitivas y emocionales. El comportamiento de los estudiantes es una de las más estudiadas. Ejemplos de las variables cuantitativas en este dominio son: resultados de consultas realizadas en un motor de búsqueda, teclas presionadas, rastreo ocular, tareas de búsqueda, esfuerzo (intentos por finalizar tareas asignadas), participación, tiempos de permanencia o de respuestas y uso de sitios web, entre otras. En el trabajo realizado por Shah, Hendahewa y González‐Ibáñez (2016) se presentan diversas métricas para evaluar el rendimiento de la búsqueda de información, en base a las distintas acciones hechas por usuarios. Tales métricas se usan con el objetivo de pronosticar la probabilidad de un usuario de tener éxito en el futuro, en base a su desempeño actual.  


\subsection{Usos de la minería de datos educacional}

%TODO: Expandir esto
%\textcite{baker2010data} desarrolló un modelo de predicción usando datos recopilados automáticamente de interacciones entre estudiantes y el \ingles{software} como variables de predicción, y después validando la precisión del modelo al ser generalizado a más estudiantes y contextos. Entonces fueron capaces de estudiar sus avances en el conjunto completo de datos.

Actualmente, la aplicación de la MDE radica en universidades, tales como Paul Smith’s College, la cual utiliza sus datos históricos para mejorar las tasas de retención de alumnos \parencite{bichsel2012analytics}. En este contexto, University of Georgia desarrolló un modelo para predecir la tasa de graduación y abandono estudiantil, el cual se alimenta en base la información recopilada \parencite{morris2005predicting}. Finalmente, la Purdue University han usado MDE para determinar que la evaluación en etapas tempranas y de forma frecuente permite cambiar los hábitos de los estudiantes con calificaciones bajo la media en cursos introductorios, en base a este trabajo, el mismo equipo de investigación desarrollo un sistema de alerta académica temprana para saber el desempeño de los estudiantes \parencite{baepler2010academic}. 

\textcite{merceron2005educational} establece cómo los algoritmos de minería de datos pueden escoger información pedagógica importante. El conocimiento obtenido ayuda a mejorar el cómo administrar la clase, como el alumno aprende, y cómo proporcionar un feedback a los alumnos. Basado en este trabajo, \textcite{abdullah2014students} realiza un sistema de predicción del rendimiento de los estudiantes basado en la actividad actual y mediciones anteriores clasificando cuales estudiantes rendirán bien y los que no. 

%Dickerson \& Hazelton (2012) exploran y ponen en práctica las técnicas de minería de datos para crear un nuevo circuito de retroalimentación hacia los profesores para la evaluación de programas y la adaptación sistemática a los cambios que requiere el sector empresarial de sus estudiantes.

%Guruler \& Istanbullu (2014) evaluan y desarrollan un enfoque basado en datos relacionados con la mejora del rendimiento de los estudiantes universitarios mediante la aplicación de técnicas incluidas en el ámbito del "Descubrimiento de Conocimiento en Bases de Datos" (conocido como KDD, las iniciales de \ingles{Knowledge Discovery in Databases}), combinado con minería de datos. 

%\textcite{sarala2015empirical} discute las aplicaciones de la minería de datos en instituciones educativas, para extraer la información útil de grandes conjuntos de datos (\ingles{datasets}), y proporciona herramientas analíticas para ver y utilizar esta información para tomar decisiones basadas en ejemplos de la vida real.

%\textcite{kalpana2014intellectual} analiza el rendimiento de los estudiantes de la facultad de ingeniería de conducted the students’ performance analysis on the graduate students’ data collected from the Villupuram college of Engineering and Technology. The data included five year period and applied clustering methods on the data to overcome the problem of low score of graduate students, and to raise students academic performance .

%\textcite{suyal2014quality} aplica técnicas de asociación y clasificación para predecir el rendimiento académico de los estudiantes applied the association and classification rule to identify the students’ performance. They mainly focused to find the students who need special attention to reduce failure rate . 

\subsection{Técnicas utilizadas en la minería de datos educacional}

%Khan \& Choi (2014) a través de un árbol de decisión y algoritmos procesan los datos de los estudiantes para calcular las posibilidades de ganar una beca en función de su grado de semestre, la ubicación del alumno en clase, la cantidad máxima y mínima de horas de crédito tomadas y permitidas y las actividades extracurriculares. 

\textcite{chen2008integrated} evalúa el rendimiento académico de estudiantes de pregrado estudiando datos académicos del Departamento de Ciencias de la Computación de National Defence Univiersity of Malaysia (NUDM) utilizando una combinación de técnicas de minería de datos, como ANN (\ingles{Artificial Neural Network}) y árboles de decisión como un método de clasificación con el que se producen ocho reglas para la identificación automática de los estilos cognitivos de los estudiantes basados en sus patrones de aprendizaje. Los hallazgos obtenidos se aplicaron para desarrollar un modelo que pueda apoyar el desarrollo de programas educativos \ingles{web}.

\textcite{moreno2009data} predice la probabilidad de que los estudiantes de acuerdo a sus registros académicos históricos fallen en un curso \ingles{online} en Moodle\footnote{https://moodle.org/} haciendo uso de las técnicas de maximización, el método de agrupamiento K-means y X-means, usando el \ingles{software} WEKA.

%TODO: Profundizar
%Dutt es un review
%\textcite{dutt2015clustering} consolida las variantes de algoritmos de clustering aplicados al contexto de MDE. Además, simplifica el diseño los sistemas que aprenden de los datos, utilizando técnicas y algoritmos de minería de datos, tales como, clustering, clasificación y predicción.

\textcite{lahtinen2005study} estudia las dificultades de aprender programación con el objetivo de crear material adecuado para introducir el curso a los estudiantes utilizando el método de agrupamiento K-means y \ingles{Hierarchical clustering} (más conocido como Ward's \ingles{clustering}, a través de este estudio se obtuvo las dificultades que sufren los estudiantes al momento de enfrentar tareas de programación. Basado en este trabajo \textcite{akinola2012data} aplica ANN para predecir el resultado de los cursos de programación en estudiantes de pregrado basados en su historial académico, los resultados de este estudio muestran que los estudiantes con un conocimiento a priori de física y matemática tienen mejor desempeño en los cursos que el resto.

%TODO: FIX THIS
\textcite{borkar2014attributes} evalúa el rendimiento de los estudiantes, donde selecciona algunos atributos mediante minería de datos, haciendo uso de una red neuronal multicapa perceptrón y usando una validación cruzada selecciona las características más influyentes, estableciendo las reglas necesarias para poder detectar las características necesarias para poder predecir el rendimiento de los estudiantes. Basado en los métodos propuestos y el mismo conjunto de datos de este trabajo, \textcite{jayakameswaraiah2014study} compara los métodos de perceptrón multicapa, Naive Bayes, SMO y J48 con el objetivo de obtener el mejor algoritmo de clasificación y predicción entre todos ellos. De todos los métodos comparados, el método de perceptrón multicapa obtuvo un \ingles{accuracy} de 75\%.


%fuente: http://www.ijiee.org/vol5/513-F1002.pdf
%[33] To analyze the web log data files of a Leaning Management System (LMS). Markov Clustering (MCL) algorithm for clustering the students‟ activity and a SimpleKMeans algorithm for clustering the courses.  The data are from the spring semester of 2009 from the Department of Information management and involve 1199 students and 39 different courses. The data are in ASCII form and are obtained from the Apache server log file

% Sacar referencias de: http://ieeexplore.ieee.org.ezproxy.usach.cl/stamp/stamp.jsp?tp=&arnumber=7684167
%Sheik and Gadage have done the analysis related to the student learning behavior by using different data mining models, namely classification, clustering, decision tree, sequential pattern mining and text mining. They used open source tools such as KNIME (Konstanz Information Miner), RAPIDMINER, WEKA, CARROT, ORANGE, RProgramming, and iDA. These tools have different compatibilities and it provided an insight into the prediction and evaluation [2]

%Mythili M S and Shanavas A R applied classification algorithms to analyze and evaluate school students performance using weka. They came with various classification algorithms, namely J48, Random Forest, Multilayer perception, IBI and decision table with the data collected from the student management system [3]. 

%Osmanbegovic and Suljic conducted a study for investigating students’ future performance in the end semester results at the University of Tuzla. They considered 11 factors and used classification model with highest accuracy for naive Bayes [5]

%Noah, Barida and Egerton conducted a study to evaluate students’ performance by grouping the grading into various classes using CGPA. They used different methods like Neural network, Regression and K-means to identify the weak performers for the purpose of performance improvement [7]. 


\subsection{Identificación de factores}
\textcite{borkar2013predicting} sugiere un método de evaluación del rendimiento de los estudiantes usando reglas asociativas de minería de datos, estimando el resultado de los estudiantes basado en la asistencia a sus cursos y su avance académico. Basado en este trabajo, \textcite{shazmeen2013performance} evalúa el rendimiento de diferentes algoritmos de clasificación y análisis predictivo, proponiendo técnicas de preprocesamiento de datos para lograr mejores resultados.

\textcite{oskouei2014predicting} identifica los factores que afectan el rendimiento de los estudiantes en diferentes países, y aplica técnicas de clasificación y predicción para mejorar la precisión de las predicciones de los resultados de los estudiantes. Los resultados muestran que los factores de género, entorno familiar, nivel de educación de los padres, y el estilo de vida afectan el rendimiento académico de los estudiantes independiente del país.

%\parencite{marquez2013predicting}

%En el trabajo de Chen et al. (2001) se hace una correlación entre el el movimiento de los ojos (medidos con un sensor de rastreo ocular) y el movimiento del ratón al navegar en la web. Dicho estudio muestra que 84% de las veces que una región de la ventana es visitada por el cursor del ratón, también es mirada por el usuario. Además el 88% de las regiones que son miradas por el usuario, tampoco son visitadas por el cursor del ratón. En síntesis, es posible afirmar que se puede utilizar la posición del ratón como un indicador de la mirada del usuario en la pantalla de forma relativamente precisa, además de ser no invasiva y ser de bajo costo.

%Tanto White & Drucker (2007) como Odijk et al. (2015) detallan que el análisis de los patrones de interacción de los usuarios dentro de ambientes de búsqueda web permiten revelar diferentes comportamientos, y por ende, diferentes estilos de procesar y asimilar la información a la hora de ejecutar tareas de búsqueda. Para realizar este análisis se debe llevar un registro de la sesión y páginas visitadas por el usuario durante la búsqueda.

%En Jiang & Ni (2016) se hace alusión a la reformulación de consultas de búsqueda web como elemento a considerar a la hora de analizar la efectividad de la búsqueda y el refinamiento que ésta alcanza a través del “ensayo y error”. Dicho análisis puede ser llevado a cabo rastreando la actividad del teclado (keystrokes) y/o la actividad del formulario de búsqueda.


%TODO: ¿Que es un buen modelo?
%FULL REFERENCE:
%“Un buen modelo cognitivo del conocimiento del estudiante debe ser capaz de predecir las diferencias en la dificultad de una tarea, o cómo es que el aprendizaje es transferido de tarea en tarea” (Koedinger et al. , 2015, pág. 339). Las investigaciones en minería de datos educacional se realizan mayoritariamente en aprendizaje online y en casos puntuales en educación superior, por lo que es limitada la información respecto a educación primaria o secundaria, sobre todo en lo que concierne a predicción de errores y fracaso escolar (Márquez-Vera et al., 2013). 
%\textcite{koedinger2015data} define que un buen modelo cognitivo de un estudiante debe ser capaz de predecir las diferencias en la dificultad de una tarea, y como el aprendizaje es transferido de tarea en tarea.

%Baradwaj and pal described data mining techniques that help in early identification of student dropouts and students who need special attention. Here they used a decision tree by using information like attendance, class test, semester and assignment marks [8]. 

%Jeevalatha, Ananthi, and Saravana Kumar presented a case study on performance analysis for placement selection for undergraduate students. They applied decision tree algorithm by considering the factors like HSC, UG marks and communication skills [9]. 

%En esta sección se presenta el estado del arte que da soporte a este trabajo, el cual comprende en primer lugar el estudio del comportamiento de estudiantes. En segundo lugar, técnicas de minería de datos y plataformas de aprendizaje de máquina aplicada al contexto educacional.

%\subsection{Plataformas de aprendizaje de máquinas aplicadas al contexto educacional}

%Cuando se aplica minería de datos en instituciones educativas, la disciplina se conoce como minería de datos educacional (MDE, desde ahora en adelante).

%La MDE es una disciplina en evolución que usa tecnologías informáticas, como son almacenes de datos y herramientas de inteligencia de negocios para descubrir tendencias y patrones sobre datos educacionales. El conocimiento que la MDE genera apoya a las autoridades de centros de educación en la toma de decisiones oportunas, y a los profesores para analizar el comportamiento y aprendizaje de sus alumnos \parencite{romero2010educational}. La disciplina se enfoca en el diseño de modelos para mejorar las experiencias del aprendizaje y la eficiencia organizacional \parencite{pandey2013decision}. El principal objetivo de la MDE es visto por diferentes investigadores como: i) modelado del estudiante, ii) modelado del dominio, iii) sistema de aprendizaje, iv) construir modelos computacionales, y v) estudiar los efectos de los recursos \parencite{merceron2005educational,kumar2015comprehensive,romero2010educational}.

%TODO: ¿En base a que criterio son relevantes?
%Las contribuciones de \textcite{romero2010educational} son las relevantes en este campo hasta la fecha. Acercan la minería de datos al contexto educativo y describe los diferentes grupos de usuarios, tipos de entornos escolares y los datos que proporcionan. Luego, exponen las tareas más típicas en el ambiente escolar que pueden resueltas a través de técnicas de minería de datos.

Tal como se muestra en los antecedentes anteriores, las investigaciones en MDE se realizan mayoritariamente en aprendizaje \ingles{online} y en casos puntuales en educación superior, por lo que es limitada la información respecto a educación básica o media, específicamente en la predicción de errores y fracaso escolar. Para mayor información de trabajos relacionados con la MDE, consultar los siguientes \ingles{reviews} \parencite{shahiri2015review,sukhija2015recent,anoopkumar2016review,dutt2017systematic}.

\begin{table}[H]
\centering
\captionabove[{Clasificación del estado del arte}]{Clasificación del estado del arte}
\begin{tabular}{lccccc}\toprule
Técnicas/Prototipo&Finalidad&P2&P3&P4\\
\midrule
\rowcolor[gray]{0.9}
Árboles de decisión									                & \y & \y & \y & \y \\
Chapter 2: Background and experimental set-up 						&    & \y & \y & \y \\
\rowcolor[gray]{0.9}
Chapter 3: Evaluating and comparing outlier-selection algorithms 	&    & \y &    &    \\
Chapter 4: Stochastic Outlier Selection 							&    &    & \y &    \\
\rowcolor[gray]{0.9}
Chapter 5: Meta-features for one-class data sets 					&    &    &    & \y \\
Chapter 6: Meta-learning for one-class classifiers 					&    &    &    & \y \\
\rowcolor[gray]{0.9}
Chapter 7: Conclusions  											& \y & \y & \y & \y \\
\bottomrule
\end{tabular}
\medskip
\par\centering Fuente: Elaboración propia, (2017)
\end{table}

