%%% File encoding is UTF8

\chapter{Desarrollo de \ingles{software}}
\label{chp:desarrollo_software}
A través del presente capítulo se expone el desarrollo de \ingles{software} de apoyo a la experimentación llevado a cabo mediante la metodología de desarrollo de \ingles{software} Desarrollo Rápido de Aplicaciones (desde ahora en adelante RAD, las iniciales de \ingles{Rapid Application Development}). En primer lugar, se presenta la metodología de desarrollo de \ingles{software} donde se define brevemente en que se basa la construcción de la herramienta. En segundo lugar, se hace una definición conceptual de los aspectos básicos que debe cumplir el \ingles{software} desarrollado. Luego, se presenta el desarrollo de \ingles{software} desde el punto de vista de los prototipos construidos a razón de la metodología ocupada. Finalmente, se describe la arquitectura construida para la herramienta. 

%\begin{tikz*}[%
%	every node/.style={rectangle,align=center,minimum height=2.25em}
%]
%	\node(req) [draw] {Инженерия требований};
%	\node(req-doc) [right=5em of req] {Спецификация требований};
%	\node(design) [below right=2em and -3em of req,draw] {Проектирование};
%	\node(arch) [right=4em of design] {Архитектура};
%	\node(impl) [below right=2em and -3em of design,draw] {Имплементация};
%	\node(sw) [right=4em of impl] {Программный продукт};
%	\node(test) [below right=2em and -3em of impl,draw] {Тестирование};
%	\node(maint) [below right=2em and -3em of test,draw] {Сопровождение};
	
%	\draw[->] (req) to (design);
%	\draw[->,dashed] (req) to (req-doc);
%	\draw[->] (design) to (impl);
%	\draw[->,dashed] (design) to (arch);
%	\draw[->] (impl) to (test);
%	\draw[->,dashed] (impl) to (sw);
%	\draw[->] (test) to (maint);
%	\draw[->,dotted] (maint) -| (req);
%\end{tikz*}

\section{Desarrollo Rápido de Aplicaciones (RAD)}
La metodología utilizada en el desarrollo de \ingles{software} es la metodología RAD (\ingles{Rapid Application Development} o Desarrollo Rápido de Aplicaciones), la cual es una metodología de desarrollo que minimiza la planificación en favor de la creación rápida de prototipos. La planificación se realiza en cada iteración, permitiendo que el \ingles{software} se desarrolle más rápido y se tenga una mayor flexibilidad con los requisitos (Martin, 1991).

Utilizando RAD la planificación del desarrollo de \ingles{software} se intercala con la construcción del \ingles{software} en sí. La falta de una amplia pre-planificación general, permite que el \ingles{software} sea implementado expeditamente y hace que sea más fácil cambiar los requisitos. Cada iteración de RAD, se compone de cuatro etapas (Martin, 1991), las cuales se explican a continuación: 

\begin{enumerate}
	\item \textbf{Etapa de definición conceptual}: También conocida como “Planeación de requerimientos”, es la fase donde se definen las funciones del negocio y los alcances de la solución.
	\item \textbf{Etapa de diseño funcional}: También conocida como “Diseño del usuario” es la fase donde se modela el sistema y sus procesos. Suelen utilizar herramientas CASE para realizar el modelado mencionado.
	\item \textbf{Etapa de desarrollo}: También conocida como etapa de “Construcción”, es cuando se ejecuta el trabajo planificado en las etapas anteriores y hace el desarrollo propio del sistema.
	\item \textbf{Etapa de despliegue}: También conocida como “Implementación” es cuando el prototipo es liberado y se entrega para la evaluación por parte del cliente. 
\end{enumerate}

\smartdiagram[circular diagram:clockwise]{Edit,
  pdf\LaTeX, Bib\TeX/ biber, make\-index, pdf\LaTeX}

\begin{center}
\smartdiagramset{circular distance=4cm,
module minimum width=2.5cm,
module minimum height=1.5cm,
arrow tip=to}
\smartdiagram[circular diagram]{Analyze,Design,Develop,Test,Support,
Define}
\end{center}

\begin{center}
\smartdiagramset{border color=none,
uniform color list=teal!60 for 4 items,
arrow style=[-stealth’,
module x sep=3.75,
back arrow distance=0.75,
}
\smartdiagram[flow diagram:horizontal]{Set up,Run,Analyse,Modify~/ Add}
\end{center}

\begin{center}
\smartdiagramset{border color=none,
set color list={blue!50!cyan,green!60!lime,orange!50!red,red!80!black},
back arrow disabled=true}
\smartdiagram[flow diagram:horizontal]{Set up,Run,Analyse,Modify~/ Add}
\end{center}

\begin{center}
\smartdiagram[sequence diagram]{Pretest,Intervención,Postest}
\end{center}

\begin{center}
\smartdiagram[sequence diagram]{Pretest,Postest}
\end{center}


\begin{minipage}[c][8cm]{\textwidth}
\centering
\smartdiagramset{
uniform color list=orange!60!yellow for 5 items,
circular final arrow disabled=true,
circular distance=2.25cm,
arrow tip=to,
arrow line width=2pt,
additions={
additional item bottom color=orange!60!yellow,
additional item border color=gray,
additional item shadow=drop shadow,
additional item offset=0.65cm,
additional arrow line width=2pt,
additional arrow tip=to,
additional arrow color=orange!60!yellow,
}
}
\smartdiagramadd[circular diagram]{
aa,bb,cc,dd,ee
}{
above of module1/Start,right of module5/End
}
\smartdiagramconnect{to-}{module1/additional-module1}
\smartdiagramconnect{-to}{module5/additional-module2}
\end{minipage}

\section{Definición conceptual}

\subsection{Requerimientos de \ingles{software}}

\subsection*{Requisitos funcionales}

\begin{description}
\item [RF1]
\item [RF2]
\item [RF3]
\item [RF4]
\item [RF5]
\end{description}

\subsection*{Requisitos no funcionales}
A continuación se presenta una lista de requerimientos no funcionales que la aplicación debe cumplir. 

\begin{description}
\item [RNF1]
\item [RNF2]
\item [RNF3]
\item [RNF4]
\item [RNF5]
\end{description}
\section{Desarrollo}

\begin{table}[H]
\centering
\captionabove[{Asociación entre tareas que componen el desarrollo y los prototipos realizados}]{Asociación entre tareas que componen el desarrollo y los prototipos realizados}
\begin{tabular}{lccccc}\toprule
Tareas/Prototipo&P1&P2&P3&P4\\
\midrule
\rowcolor[gray]{0.9}
Chapter 1: Introduction 											& \y & \y & \y & \y \\
Chapter 2: Background and experimental set-up 						&    & \y & \y & \y \\
\rowcolor[gray]{0.9}
Chapter 3: Evaluating and comparing outlier-selection algorithms 	&    & \y &    &    \\
Chapter 4: Stochastic Outlier Selection 							&    &    & \y &    \\
\rowcolor[gray]{0.9}
Chapter 5: Meta-features for one-class data sets 					&    &    &    & \y \\
Chapter 6: Meta-learning for one-class classifiers 					&    &    &    & \y \\
\rowcolor[gray]{0.9}
Chapter 7: Conclusions  											& \y & \y & \y & \y \\
\bottomrule
\end{tabular}
\medskip
\par\centering Fuente: Elaboración propia, (2017)
\end{table}


\subsection{Prototipo 1}

\subsection{Prototipo 2}

\subsection{Prototipo 3}

\subsection{Prototipo 4}
\section{Arquitectura y tecnologías}
\section{Resumen}
\label{sec:desarrollo_resumen}
Este capítulo presentó el proceso seguido durante la creación de la herramienta de \ingles{software} a medida implicada en este trabajo. Lo anterior mediante el seguimiento de una metodología ágil basada en RAD que comprendió la entrega de una serie de prototipos de \ingles{software} incrementales, que fueron implementando paulatinamente cada una de las funciones necesarias para cubrir los requerimientos derivados por el diseño experimental de este estudio. 

En este proceso de desarrollo se definió conceptualmente la plataforma, exponiendo los requerimientos funcionales y no funcionales a cubrir. A partir de esto se inició un proceso iterativo de diseño y desarrollo donde se implementaron de manera progresiva las distintas funcionalidades de la plataforma, realizando un continuo proceso de retroalimentación entre el profesor guía y el tesista, para una constante refinación del \ingles{software} en términos de estética y funcionalidad en cada uno de los prototipos. 

El desarrollo completo de la plataforma comprendió la entrega de cuatro prototipos incrementales, generando como resultado un \ingles{software} 

%desarrollado en Android que por un lado posibilita la interacción del usuario con objetos de información digital (imágenes), por medio de la interfaz tradicional y la táctil, y que por otro lado sirve a un investigador como herramienta de apoyo a la evaluación de la interacción del usuario, cumpliendo de esta manera con las características estipuladas por el diseño experimental.  
%\input{02_Chapters/04_Desarrollo_Software/04_.tex}