\section{Desarrollo Rápido de Aplicaciones (RAD)}
La metodología utilizada en el desarrollo de \ingles{software} es la metodología RAD (\ingles{Rapid Application Development} o Desarrollo Rápido de Aplicaciones), la cual es una metodología de desarrollo que minimiza la planificación en favor de la creación rápida de prototipos. La planificación se realiza en cada iteración, permitiendo que el \ingles{software} se desarrolle más rápido y se tenga una mayor flexibilidad con los requisitos (Martin, 1991).

Utilizando RAD la planificación del desarrollo de \ingles{software} se intercala con la construcción del \ingles{software} en sí. La falta de una amplia pre-planificación general, permite que el \ingles{software} sea implementado expeditamente y hace que sea más fácil cambiar los requisitos. Cada iteración de RAD, se compone de cuatro etapas (Martin, 1991), las cuales se explican a continuación: 

\begin{enumerate}
	\item \textbf{Etapa de definición conceptual}: También conocida como “Planeación de requerimientos”, es la fase donde se definen las funciones del negocio y los alcances de la solución.
	\item \textbf{Etapa de diseño funcional}: También conocida como “Diseño del usuario” es la fase donde se modela el sistema y sus procesos. Suelen utilizar herramientas CASE para realizar el modelado mencionado.
	\item \textbf{Etapa de desarrollo}: También conocida como etapa de “Construcción”, es cuando se ejecuta el trabajo planificado en las etapas anteriores y hace el desarrollo propio del sistema.
	\item \textbf{Etapa de despliegue}: También conocida como “Implementación” es cuando el prototipo es liberado y se entrega para la evaluación por parte del cliente. 
\end{enumerate}

\smartdiagram[circular diagram:clockwise]{Edit,
  pdf\LaTeX, Bib\TeX/ biber, make\-index, pdf\LaTeX}

\begin{center}
\smartdiagramset{circular distance=4cm,
module minimum width=2.5cm,
module minimum height=1.5cm,
arrow tip=to}
\smartdiagram[circular diagram]{Analyze,Design,Develop,Test,Support,
Define}
\end{center}

\begin{center}
\smartdiagramset{border color=none,
uniform color list=teal!60 for 4 items,
arrow style=[-stealth’,
module x sep=3.75,
back arrow distance=0.75,
}
\smartdiagram[flow diagram:horizontal]{Set up,Run,Analyse,Modify~/ Add}
\end{center}

\begin{center}
\smartdiagramset{border color=none,
set color list={blue!50!cyan,green!60!lime,orange!50!red,red!80!black},
back arrow disabled=true}
\smartdiagram[flow diagram:horizontal]{Set up,Run,Analyse,Modify~/ Add}
\end{center}

\begin{center}
\smartdiagram[sequence diagram]{Pretest,Intervención,Postest}
\end{center}

\begin{center}
\smartdiagram[sequence diagram]{Pretest,Postest}
\end{center}


\begin{minipage}[c][8cm]{\textwidth}
\centering
\smartdiagramset{
uniform color list=orange!60!yellow for 5 items,
circular final arrow disabled=true,
circular distance=2.25cm,
arrow tip=to,
arrow line width=2pt,
additions={
additional item bottom color=orange!60!yellow,
additional item border color=gray,
additional item shadow=drop shadow,
additional item offset=0.65cm,
additional arrow line width=2pt,
additional arrow tip=to,
additional arrow color=orange!60!yellow,
}
}
\smartdiagramadd[circular diagram]{
aa,bb,cc,dd,ee
}{
above of module1/Start,right of module5/End
}
\smartdiagramconnect{to-}{module1/additional-module1}
\smartdiagramconnect{-to}{module5/additional-module2}
\end{minipage}
