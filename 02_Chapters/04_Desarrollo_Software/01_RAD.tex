\section{Desarrollo Rápido de Aplicaciones (RAD)}
La metodología utilizada en el desarrollo de \ingles{software} es la metodología RAD (\ingles{Rapid Application Development} o Desarrollo Rápido de Aplicaciones), la cual es una metodología de desarrollo que minimiza la planificación en favor de la creación rápida de prototipos. La planificación se realiza en cada iteración, permitiendo que el \ingles{software} se desarrolle más rápido y se tenga una mayor flexibilidad con los requisitos (Martin, 1991).

Utilizando RAD la planificación del desarrollo de \ingles{software} se intercala con la construcción del \ingles{software} en sí. La falta de una amplia pre-planificación general, permite que el \ingles{software} sea implementado expeditamente y hace que sea más fácil cambiar los requisitos. Cada iteración de RAD, se compone de cuatro etapas (Martin, 1991), las cuales se explican a continuación: 

\begin{enumerate}
	\item \textbf{Etapa de definición conceptual}: También conocida como “Planeación de requerimientos”, es la fase donde se definen las funciones del negocio y los alcances de la solución.
	\item \textbf{Etapa de diseño funcional}: También conocida como “Diseño del usuario” es la fase donde se modela el sistema y sus procesos. Suelen utilizar herramientas CASE para realizar el modelado mencionado.
	\item \textbf{Etapa de desarrollo}: También conocida como etapa de “Construcción”, es cuando se ejecuta el trabajo planificado en las etapas anteriores y hace el desarrollo propio del sistema.
	\item \textbf{Etapa de despliegue}: También conocida como “Implementación” es cuando el prototipo es liberado y se entrega para la evaluación por parte del cliente. 
\end{enumerate}