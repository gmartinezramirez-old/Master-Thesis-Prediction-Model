\section{Resumen}
\label{sec:desarrollo_resumen}
Este capítulo presentó el proceso seguido durante la creación de la herramienta de \ingles{software} a medida implicada en este trabajo. Lo anterior mediante el seguimiento de una metodología ágil basada en RAD que comprendió la entrega de una serie de prototipos de \ingles{software} incrementales, que fueron implementando paulatinamente cada una de las funciones necesarias para cubrir los requerimientos derivados por el diseño experimental de este estudio. 

En este proceso de desarrollo se definió conceptualmente la plataforma, exponiendo los requerimientos funcionales y no funcionales a cubrir. A partir de esto se inició un proceso iterativo de diseño y desarrollo donde se implementaron de manera progresiva las distintas funcionalidades de la plataforma, realizando un continuo proceso de retroalimentación entre el profesor guía y el tesista, para una constante refinación del \ingles{software} en términos de estética y funcionalidad en cada uno de los prototipos. 

El desarrollo completo de la plataforma comprendió la entrega de cuatro prototipos incrementales, generando como resultado un \ingles{software} 

%desarrollado en Android que por un lado posibilita la interacción del usuario con objetos de información digital (imágenes), por medio de la interfaz tradicional y la táctil, y que por otro lado sirve a un investigador como herramienta de apoyo a la evaluación de la interacción del usuario, cumpliendo de esta manera con las características estipuladas por el diseño experimental.  