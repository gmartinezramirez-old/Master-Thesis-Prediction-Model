\section{Iteraciones y prototipos}

En esta oportunidad el rol de cliente recayó en el profesor guía de esta tesis, el Dr. Roberto González, quien decidió que aceptar, descartar o cambiar de cada uno de los prototipos de \ingles{software} que comprendió este desarrollo. 

A continuación, se procede a explicar los cuatro prototipos implicados en el desarrollo, por temas de organización del contenido en el informe se habla de los aspectos finales del diseño y desarrollo del \ingles{software} para cada prototipo, entendiendo que este es un proceso iterativo e incremental donde se desarrolla en base a los avances de prototipos anteriores tal como se expone en la \tab{table:prototipos}.

\begin{table}[H]
\centering
\captionabove[{Asociación entre tareas que componen el desarrollo y los prototipos realizados}]{Asociación entre tareas que componen el desarrollo y los prototipos realizados}
\begin{tabular}{lccccc}\toprule
Tareas/Prototipo&P1&P2&P3&P4\\
\midrule
\rowcolor[gray]{0.9}
RF1										& \y & \y & \y & \y \\
RF2 						&    & \y & \y & \y \\
\rowcolor[gray]{0.9}
RF3 	&    & \y &    &    \\
RF4 							&    &    & \y &    \\
\rowcolor[gray]{0.9}
RF5 					&    &    &    & \y \\
RNF1 					&    &    &    & \y \\
\rowcolor[gray]{0.9}
RNF2  											& \y & \y & \y & \y \\
\bottomrule
\end{tabular}
\medskip
\par\centering Fuente: Elaboración propia, (2017)
\label{table:prototipos}
\end{table}


\subsection{Prototipo 1}

\subsection{Prototipo 2}

\subsection{Prototipo 3}

\subsection{Prototipo 4}