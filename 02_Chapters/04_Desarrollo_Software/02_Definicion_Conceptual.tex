\section{Definición conceptual}

\subsection{Requerimientos de \ingles{software}}
A continuación se presentan los requerimientos de \ingles{software}. Estos requerimientos pueden ser de dos tipos (Sommerville, 2005): requerimientos funcionales (acciones que puede realizar el sistema) y requerimientos no funcionales (condiciones de como se deberían realizar dichas acciones).

Para la nomenclatura de los requisitos, se utilizará la siguiente convención: \textbf{XX-nn} donde \textbf{XX} corresponde a el tipo de requisito (RF para requerimientos funcionales y RNF para los no funcionales), y \textbf{nn} corresponde al número del requerimiento.

\subsubsection*{Requisitos funcionales}
A continuación se presenta una lista de requerimientos funcionales (RF) que la aplicación debe cumplir. 


\begin{itemize}
\item \textbf{RF-01}:
\item \textbf{RF-02}:
\item \textbf{RF-03}:
\item \textbf{RF-04}:
\item \textbf{RF-05}:
\end{itemize}

\subsubsection*{Requisitos no funcionales}
A continuación se presenta una lista de requerimientos no funcionales (RNF) que la aplicación debe cumplir. 

\begin{itemize}
\item \textbf{RNF-01}:
\item \textbf{RNF-02}:
\item \textbf{RNF-03}:
\item \textbf{RNF-04}:
\item \textbf{RNF-05}:
\end{itemize}

\subsection{Análisis de factibilidad}
A partir del desglose de los requerimientos, se lleva a cabo un análisis de factibilidad por componentes, esto a partir de un contraste con las tecnologías conocidas por el desarrollador y las alternativas tecnológicas en el estado del arte. A continuación se presentan los resultados de este análisis.

\subsubsection*{\ingles{Backend}}


\subsubsection*{\ingles{Frontend}}


\subsubsection*{Base de datos}

\subsubsection*{\ingles{Frameworks} de aprendizaje automático}


\subsubsection*{Estimación de los requisitos mínimos}
En base a las herramientas indicadas en los puntos anteriores, se hace la estimación de requisitos de \ingles{hardware} y \ingles{software} indicada a continuación para ejecutar el \ingles{software} construido como cliente y servidor.