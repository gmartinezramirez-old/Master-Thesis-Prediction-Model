%%% File encoding is UTF8
%%% You can use special characters just like ä,ü and ñ

% Chapter without numbering but with appearance in the Table of Contents
% \addchap is a command from KOMA-Script
%\addchap*{Abstract}

\chapter*{\centerline{Resumen}} \vspace{-6em}

%Only in propuesta
%El presente documento corresponde a la propuesta de tesis para la carrera de Ingeniería Civil en Informática y Magister en Ingeniería Informática, cuyo título es ``Modelo predictivo del desempeño de búsqueda de información en línea en estudiantes de educación básica''. A continuación se introduce el problema a resolver a lo largo del trabajo.

Durante la última década, debido a los rápidos avances de las tecnologías de la información y comunicación ha aumentado la cantidad de recursos digitales en Internet, la diversidad de fuentes de información, además, se ha facilitado el acceso a estos. Asimismo, las búsquedas \ingles{web} han pasado a ser parte de las tareas comunes que realizan los estudiantes de los planteles educativos. Considerando la diversidad de fuentes de información y tipos de recursos en línea, resulta necesario desarrollar competencias informacionales durante el proceso de formación en los distintos niveles educativos (primaria, secundaria y universitaria).

En el marco del proyecto iFuCo (\ingles{Enhancing learning and teaching future competences of online inquiry in multiple domains}), formado por investigadores de Chile y Finlandia, el cual desea investigar y modelar los comportamientos y competencias de investigación en línea de estudiantes de enseñanza básica, se propone la construcción de un modelo de predicción del desempeño de búsqueda de información en línea en estudiantes de educación básica el cual se vaya perfeccionando a través del registro de datos históricos y que de una retroalimentación de forma continua.


%La investigación será guiada por la metodología KDD con el fin de descubrir patrones en los datos que permitan la creación de un modelo de predicción del comportamiento de búsqueda. Además, para apoyar el proceso de investigación, se desarrollará una plataforma que funcione como extensión de la plataforma NEURONE (\ingles{oNlinE inqUiry expeRimentatiON systEm}). La plataforma propuesta alimentará y perfeccionará el modelo de predicción y entregará predicciones en tiempo real. Esta plataforma se guiará bajo la metodología RAD (\ingles{Rapid Application Development}) la cual se orienta a un desarrollo iterativo e incremental para la rápida construcción de prototipos de \ingles{software}.

\par\noindent
{\bfseries Palabras Claves\/}: Alfabetización informacional, Competencias de investigación en línea, Comportamiento de estudiantes, Minería de datos, Modelos de clasificación.

%En propuesta no
%\newpage
%\chapter*{\centerline{Abstract}} \vspace{-6em}
%During the last decade, due to rapid advances in information and communication technologies, the number of digital resources on the Internet has increased, and the diversity of information sources has also facilitated access to information. In addition, web searches have become part of the common tasks performed by students in schools. Considering the diversity of sources of information and types of online resources, it is necessary to develop informational competences during the training process at different levels of education (primary, secondary and university/college).

%In the framework of the iFuCo project, which consists of researchers from Chile and Finland, who wishes to investigate and model online research behaviors and competencies of basic education students, we propose the construction of a prediction model of information search performance In line in students of basic education that is perfected through the registry of historical data and that of a feedback of continuous form.



%\par\noindent
%{\bfseries Keywords\/}: Data mining, Machine learning, Classification models.