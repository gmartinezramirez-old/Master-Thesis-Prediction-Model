%%% File encoding is UTF8
%%% You can use special characters just like ä,ü and ñ

% ##############################################
% Start: Table of Contents (TOC) Customization
% ##############################################
%

% Level for numbered captions
\setcounter{secnumdepth}{5}

% Level of chapters that appear in Table of Contents
\setcounter{tocdepth}{5} % bis wohin ins Inhaltsverzeichnis aufnehmen
% -2 no caption at all
% -1 part
% 0  chapter
% 1  section    
% 2  subsection 
% 3  subsubsection
% 4  paragraph
% 5  subparagraph

% KOMA-Script code to adjust TOC
% Applying the color 'myColorMainA' which is defined in the main file (MainFile.tex)
\makeatletter
%Dont use color
\addtokomafont{chapterentrypagenumber}{\color{myColorMainA}}
\addtokomafont{chapterentry}{\color{myColorMainA}}
\makeatother

\renewcaptionname{spanish}{\contentsname}{\hfill\bfseries\Large Tabla de Contenidos \hfill} % Table of contents in spanish
\renewcaptionname{spanish}{\listfigurename}{\hfill\bfseries\Large \'Indice de Figuras \hfill}    %Figures
\renewcaptionname{spanish}{\listtablename}{\hfill\bfseries\Large \'Indice de Tablas \hfill}        %Tables

%Funciona
\makeatletter
\let\oldaddchaptertocentry\addchaptertocentry
\renewcommand{\addchaptertocentry}[2]{%
	\oldaddchaptertocentry{}{\@chapapp{} #1. #2}}
\let\oldaddparttocentry\addparttocentry
\renewcommand{\addparttocentry}[2]{%
	\oldaddchaptertocentry{}{\partname{} #1. #2}}

\RedeclareSectionCommand[tocindent=5.5em]{section}
\RedeclareSectionCommand[tocindent=7.8em]{subsection}

\makeatother

\makeatletter
\newcommand{\chapterwithfigures}{\addxcontentsline{lof}{chapteratlist}[{\thechapter}]{\scr@ds@tocentry}}%
\makeatother

%
% #######################
% End: Table of Contents (TOC) Customization
% #######################

% ##############################################
% Start: Floating Object Customization
% ##############################################
%

% Extended support for catioons of figures and tables etc.
% CTAN: http://www.ctan.org/pkg/caption
\usepackage[font={small}
		  %, labelfont={bf,sf}
		  , labelfont={sf}
		  , format=hang % try plain or hang
		  , margin=5pt
]{caption}
%

\newcommand{\englishword}[1]{\textit{#1}}
\newcommand{\ingles}[1]{\textit{#1}}
\newcommand{\producto}[1]{\textit{#1}}
\newcommand{\tecnologia}[1]{\textit{#1}}

\newcommand*{\captionsource}[2]{%
	\caption[{#1}]{#1}
	\par\centering\small{Fuente: #2.}
}

\newcommand*{\captiontable}[2]{%
	\captionabove[{#1}]{#1 \par\centering \small{Fuente: #2.}}
}

% Make Uppecase section title
\addtokomafont{section}{\MakeUppercase}

% Tables with PGFPlots
\usepackage{pgfplotstable} % Generates table from .csv
\usepgfplotslibrary{colorbrewer}
\usepgfplotslibrary{statistics}

\pgfplotsset{cycle list/Dark2}
\pgfplotsset{cycle list/Dark2-5}

% Table: htb: place here, top, or bottom, BUT NEVER at other page
%Args: 
% 1: Caption
% 2: Source
% 3: CSV File (separated by semicolons ;)
% 4: Label command
\newcommand{\insertTableFile}[4]{
	\begin{table}[htb]
	\centering
	\captiontable{#1}{#2}
	\pgfplotstabletypeset[
	every even row/.style={before row={\rowcolor[gray]{0.9}}},
	every head row/.style={before row=\toprule,after row=\midrule},
	every last row/.style={after row=\bottomrule},
	col sep=semicolon, 
	string type]{#3}
	#4 % Label command		
	\end{table}
}

\newcommand{\insertTableFileMoreText}[4]{
	\begin{table}[htb]
	\centering
	\captiontable{#1}{#2}
	\pgfplotstabletypeset[
	columns={Actividad;Duración (HH)},
	columns type=r,
	every even row/.style={
		before row={\rowcolor[gray]{0.9}}},
	every head row/.style={
		before row=\toprule,after row=\midrule},
	every last row/.style={
		after row=\bottomrule},
	col sep=semicolon, 
	string type]{#3}
	#4 % Label command		
	\end{table}
}


% Arg 1: X label
% Arg 2: Y label
% Arg 3: Filename
% Arg 4: Caption
% Arg 5: Source
% Arg 6: Label

\newcommand{\plotFile}[3]{
	\pgfplotsset{cycle list/Dark2}
	\begin{figure}[htb]
	\centering
	\begin{tikzpicture}
	\begin{axis}[
	xlabel={#1},
	ylabel={#2},
	xmin=0,
	ymin=0,
	ymax=1,
	legend style = {
		legend pos = outer north east,
	},
	%cycle list name=Dark2,
	width=0.7\textwidth
	]
	\legend{Legend 1}
	\pgfplotstableread{#3}\leg1;
	\addplot+ [
	cycle list name=Dark2,
	mark=square*
	] table []{\leg1};
	\end{axis}
	\end{tikzpicture}
	%\captionsource{#4}{#5}
	%#6	
	\end{figure}

}

\newcommand{\correcion}[1]{\hl{#1}}
\newcommand{\currentYear}{2017}
\newcommand{\fuentePropia}{Elaboración propia, (\currentYear)}
\newcommand{\traduccionlibre}{Traducción libre.}


%% References
\newcommand{\tab}[1]{Tabla~\ref{#1}}
\newcommand{\tabla}[1]{Tabla~\ref{#1}}
\newcommand{\fig}[1]{Figura~\ref{#1}}
\newcommand{\alg}[1]{Algoritmo~\ref{#1}}
\newcommand{\eq}[1]{Ecuación~\ref{#1}}


% #######################
% End: Floating Object Customization
% #######################

% ##############################################
% Start: Headings Customization
% ##############################################
%

% KOMA-Script code to customize the headings
% Applying the color 'myColorMainA' which is defined in the main file (MainFile.tex)
%\addtokomafont{chapter}{\color{myColorMainA}}
%\addtokomafont{section}{\color{myColorMainA}}
%\addtokomafont{subsection}{\color{myColorMainA}}
%\addtokomafont{subsubsection}{\color{myColorMainA}}
%\addtokomafont{paragraph}{\color{myColorMainA}}
%\addtokomafont{subparagraph}{\color{myColorMainA}}

% #######################
% End: Headings Customization
% #######################
% Definition of commands for fancy tables
%\newcommand{\myrowcolour}{\rowcolor[gray]{0.9}}
%\newcolumntype{L}{>{\raggedright\arraybackslash}m{10cm}}