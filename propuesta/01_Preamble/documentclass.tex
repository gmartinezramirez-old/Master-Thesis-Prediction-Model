%%% File encoding is UTF8
%%% You can use special characters just like ä,ü and ñ

% KOMA-Script class 'scrbook'
% Link to the documentation: 
% German: http://mirrors.ctan.org/macros/latex/contrib/koma-script/doc/scrguide.pdf
% English: http://mirrors.ctan.org/macros/latex/contrib/koma-script/doc/scrguien.pdf
% CTAN: http://www.ctan.org/pkg/koma-script
% Author of the KOMA-Script family is Markus Kohm
\documentclass[paper=letter % Letter is "carta"
			 , fontsize=10pt % Arial 10 for all text.
			 , headings=big
			 , parskip=half-
			%, noparskip % No par skip. If yes park skip the command is parskip
			%, numbers=noendperiod % 2.3.1 vs 2.3.1. (no dot after the last chapter number)
			 , numbers=endperiod
	         , twoside=false
			% , toc=bibliography % Bibliography appears in Table of Contents (without a number)
			% , toc=listof % List of Figures and List of Tables appear in Table of Contents
			% , chapterprefix=true
			 , chapterprefix=false
			% , bibliography=totoc % The bibliography will have an entry at the table of contents but no number.
			 , numbers=noenddot
			 , titlepage=simple
			 , headings=onelinechapter
			 , version=last % Use latest version of the KOMA-Script
			 , final
]{scrbook}

% TODO: Revisar las separaciones
\RedeclareSectionCommand[
afterskip=\parskip]{chapter}
\RedeclareSectionCommand[
beforeskip=\parskip
, afterskip=0.1\parskip]{section}
\RedeclareSectionCommand[
beforeskip=1.1\parskip plus 3mm
, afterskip=0.1\parskip]{subsection}


\KOMAoption{bibliography}{leveldown}

\newcommand{\mychapter}[1]{%
	\chapter{#1}
	\addcontentsline{lof}{chapter}{%
		Figuras del Capítulo \thechapter: #1 \vspace{10pt}
	}
}

\addtokomafont{section}{\large}
\addtokomafont{subsection}{\normalsize}  %1.1.1 Minus size: 10 % normalsize is 10

%REVISAR
\usepackage[compact]{titlesec}

% Titles in the chapters, not in TOC.
\titleformat{\chapter}
{\bfseries\Large\vspace*{-4.0cm}}	% Formato título
{	% Contenido de la etiqueta
	\filright
	\Large\MakeUppercase\chaptertitlename\ \thechapter.\ 
}
{0pt} % Espacio mínimo entre etiqueta y cuerpo
{\filright\MakeUppercase} % Código que precede al cuerpo del título
[\vspace{1.5pt}] % Margen de 1.5pt

\titleformat{\section}
{\bfseries\large\vspace{2pt}}
{\large\MakeUppercase\thesection\ \vspace{2pt} } % 3 espacios luego del titulo de una seccion
{0pt}
{\MakeUppercase}
[\vspace*{0.5cm}]

\titleformat{\subsection}
{\bfseries\normalsize\vspace{2pt}} % 3 espacios luego del titulo de una seccion
{\normalsize\thesubsection\ }
{0pt}
{\vspace*{0.5cm}}

\titleformat{\subsubsection}
{\itshape\normalsize\vspace{1.0cm}}
{\itshape\thesubsubsection\ }
{0pt}
{\vspace*{0.5cm}\itshape}

\titlespacing*{\chapter} {0pt}{85pt}{20pt} 
\titlespacing*{\section} {0pt}{6.5ex plus 1ex minus .2ex}{2.3ex plus .2ex}
\titlespacing*{\subsection} {0pt}{6.5ex plus 1ex minus .2ex}{2.3ex plus .2ex}
\titlespacing*{\subsubsection}{0pt}{3.25ex plus 1ex minus .2ex}{1.5ex plus .2ex}
\titlespacing*{\paragraph} {0pt}{3.25ex plus 1ex minus .2ex}{1em}
\titlespacing*{\subparagraph} {\parindent}{3.25ex plus 1ex minus .2ex}{1em}
