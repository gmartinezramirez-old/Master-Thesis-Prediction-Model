\chapter{Metodología, Herramientas, y Ambiente de Desarrollo}
\label{chp:metodologia-herramientas-ambiente}

\section{Metodología a usar}
\label{sec:metodologia}
El presente proyecto presenta una componente de investigación y desarrollo de \textit{software} (I+D). Esto debido a la relación que existe entre ambas componentes, la investigación necesita una herramienta de software de apoyo que permita recibir los datos de NEURONE, alimentar el modelo de predicción y que permita al usuario interactuar con resultados de la predicción a realizar.

La componente de investigación del proyecto será guiada por la metodología Descubrimiento de Conocimiento en Base de Datos (\textit{Knowledge Discovery in Databases}, KDD desde ahora en adelante), mientras que la componente de desarrollo de \textit{software} será guiada por la metodología \textit{Rapid Application Development} (RAD, desde ahora en adelante) \cite{martin1991rapid}. A continuación, se explica el uso de ambas metodologías en el trabajo propuesto. Para mayor información sobre las metodologías a ocupar, ver ANEXO B. DESCRIPCIÓN Y USO DE LAS METODOLOGÍAS.

\subsection{Metodología usada en la investigación}
\label{sub:metodologia-investigacion}
Respecto a la componente de investigación, esta será guiada bajo la metodología KDD, la cual se define como “un proceso no trivial de identificar patrones en los datos que sean válidos, novedosos, potencialmente útiles y finalmente comprensibles” \cite{fayyad1996data}. En primer lugar, se seleccionan y limpian los datos que se deben extraer para poder realizar el modelado del comportamiento de búsqueda. Luego, se transforman los datos y se realiza minería de datos sobre ellos para buscar los patrones de interés que pueden expresarse como un modelo o que expresen dependencia de los datos. Finalmente, se identifican los patrones realmente interesantes que representan el conocimiento, usando diferentes técnicas, incluyendo análisis estadísticos y lenguajes de consultas para posteriormente interpretar los datos obtenidos.

\subsection{Metodología usada para el desarrollo de \textit{software}}
\label{sub:metodologia-investigacion}
Respecto a la componente de desarrollo de software, se recurre a un proceso de desarrollo inspirado en la metodología RAD, la cual minimiza la planificación en favor de la creación rápida de prototipos. La planificación se realiza en cada iteración, permitiendo que el software se desarrolle más rápido y se tenga mayor flexibilidad con los requisitos \cite{mcconnell1996rapid}.

\section{Herramientas de desarrollo}
\label{sec:herramientas}
Las herramientas a utilizar en el trabajo de tesis, se dividen tanto en \textit{hardware} como en \textit{software}.

\begin{itemize}
\item \textit{Hardware}: 
\item \textit{Software}:
\end{itemize}

\section{Ambiente de trabajo}
El ambiente de desarrollo del presente proyecto será tanto en el domicilio particular del candidato a tesista, como también en el Departamento de Ingeniería Informática de la Universidad de Santiago de Chile, específicamente, en el laboratorio de postgrado de dicho departamento.

Después de finalizar cada iteración, el \textit{feedback} a este proyecto es ofrecido por miembros del equipo de investigación del proyecto iFuCo, y el profesor guía, quien, además brinda apoyo en los aspectos tecnológicos y metodológicos. Finalmente, el equipo de desarrollo de este trabajo de título es unipersonal, con colaboración en fundamentos teóricos de otros tesistas involucrados en el proyecto. 

\begin{figure}[htb]
	\centering
	\includegraphics[height=5cm]{example-image-b} 
	\captionsource{A cow licking its nose. Usage with permission of the photographer \textsc{Nicole Barth}}{Obtenido de \url{www.flickr.com/photos/46311827@N07/14885545396}, (2017)}
	\label{fig:CowLickingNose}
\end{figure}