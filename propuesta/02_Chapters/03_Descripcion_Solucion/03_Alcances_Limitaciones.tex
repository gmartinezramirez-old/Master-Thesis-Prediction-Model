\section{Alcances y limitaciones de la solución}
\label{sec:alcances}
Los modelos se construyen a partir de un conjunto de datos específicos, estos datos tienen su propio contexto y origen que limitan la generalización de los modelos a construir. A continuación, se describen las principales limitaciones y alcances de la solución.

\begin{enumerate}
	\item El curso de alfabetización informacional y sus respectivos registros de datos, pertenecen al proyecto iFuCo [24], el cual es un trabajo colaborativo entre universidades de Finlandia (University of Tampere, University of Jyväskylä y University of Turku) y de Chile (Universidad de Santiago de Chile y Pontificia Universidad Católica de Chile). 
	\item Los registros de datos provienen de un estudio enmarcado en un curso de alfabetización en información, aplicado al área de Ciencia y Ciencias Sociales, en ambos países.
	\item Los datos son recolectados y almacenados por un sistema externo llamado NEURONE (\ingles{oNlinE inqUiry expeRimentatiON systEm}), trabajo de memoria de un estudiante de la carrera de Ingeniería de Ejecución en Computación e Informática de la Universidad de Santiago de Chile [25].
	\item La solución funciona como un sistema predictor del desempeño de estudiante en la búsqueda de información, sin ofrecer acciones correctivas en caso de bajo desempeño.
\end{enumerate}