\section{Características de la solución}
\label{sec:caracteristicas-solucion}

La solución consiste en una plataforma de aprendizaje de máquina y predicción del rendimiento de los estudiantes en tiempo real, en donde a través de la información obtenida en la plataforma NEURONE se crea un modelo de clasificación y predicción del rendimiento de los estudiantes de enseñanza básica en un curso de alfabetización informacional, específicamente en el tema de investigaciones en línea\footnote{Traducción libre} (\ingles{online inquiry}). 

Los datos son recopilados y almacenados por NEURONE, estos datos provienen de registros del proceso de buscar información en línea en un sistema cerrado, los cuales son: historial de navegación, consultas realizadas, movimientos del \ingles{mouse}, escritura por teclado, número de \ingles{clicks} y tiempos de permanencia en páginas \ingles{web}. Además, se conoce con anticipación los documentos y párrafos ideales a seleccionar por parte de los estudiantes.

La información de los estudiantes y el resultado de las evaluaciones definen una buena o mala navegación. La plataforma en particular debe predecir en tiempo real el desempeño de los estudiantes a partir de su comportamiento de búsqueda de información actual.

La plataforma propuesta hará uso de Tensorflow, la cual se conectará con el sistema NEURONE, funcionando como una extensión del mismo, consultando su base de datos, alimentando y perfeccionando el modelo. El ciclo de construcción, evaluación y optimización del modelo se describe en la Figura 3.1, donde el modelo está en una continua optimización.