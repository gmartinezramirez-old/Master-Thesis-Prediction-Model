\section{Características de la solución}
\label{sec:caracteristicas-solucion}
La solución consiste en incorporar un módulo en NEURONE \parencite{gonzalez2017neurone} que clasifique y prediga de forma continua el desempeño de búsqueda de los estudiantes de enseñanza básica en un curso de alfabetización informacional, específicamente en el tema de investigaciones en línea\footnote{\traduccionlibre} (\ingles{online inquiry}). 

Los datos son recopilados y almacenados por NEURONE, estos datos provienen de registros del proceso de búsqueda de información en línea en un sistema cerrado, los cuales son: historial de navegación, consultas realizadas, movimientos del \ingles{mouse}, escritura por teclado, número de \ingles{clicks} y tiempos de permanencia en páginas \ingles{web}. Además, se conoce con anticipación los documentos y párrafos ideales a seleccionar por parte de los estudiantes.

El módulo propuesto hará uso de Apache Spark\footnote{https://spark.apache.org/}, el cual es un \ingles{framework} de código abierto para el procesamiento de datos masivos, el cual incluye librerías de minería de datos y aprendizaje de máquina. Este módulo se conectará con el sistema NEURONE, funcionando como una extensión del mismo, consultando su base de datos, alimentando y perfeccionando el modelo. 

El ciclo de construcción, evaluación y optimización del modelo se ilustra en la Figura \ref{fig:ml-pipeline}, donde a través de los datos históricos obtenidos de NEURONE construye el modelo, lo evalúa y lo optimiza en un proceso continuo, entregando como resultado la clasificación del desempeño de búsqueda y prediciendo de forma continua a partir del comportamiento actual de búsqueda de información del estudiante.


%Fuente: https://tex.stackexchange.com/questions/4338/correctly-scaling-a-tikzpicture
\begin{figure}[H]
	\centering
	%FIX: dont compile
	%\scalebox{0.8}{%\centering\makebox[\textwidth]{
\begin{tikzpicture}[node distance=3cm, auto]

    % Place nodes
    \node[label={Estudiante},charlie,monitor,minimum size=1.3cm] (estudiante);
    \node[block, right of=estudiante] (data-ingestion) {Entrada de datos};
    \node[block, right of=data-ingestion] (data-clean) {Transformación de datos};
    \node[block, right of=data-clean] (model-training) {Entrenamiento del modelo};
    \node[block, right of=model-training] (model-testing) {Evaluación del modelo};
    \node[block, right of=model-testing] (prediction) {Predicción};

\draw[line] (estudiante) -- (data-ingestion)
            (data-ingestion) -- (data-clean)
            (data-clean) -- (model-training)
            (model-training) -- (model-testing)
            (model-testing) -- (prediction);

\path[line,dashed] 
   (model-testing.south) -- +(0,-.35) -| (model-training.south)
   (prediction.south) -- +(0,-.60) -| (data-ingestion.south);

\end{tikzpicture}
%}
	%\resizebox{1.0\textwidth}{!}{%\centering\makebox[\textwidth]{
\begin{tikzpicture}[node distance=3cm, auto]

    % Place nodes
    \node[label={Estudiante},charlie,monitor,minimum size=1.3cm] (estudiante);
    \node[block, right of=estudiante] (data-ingestion) {Entrada de datos};
    \node[block, right of=data-ingestion] (data-clean) {Transformación de datos};
    \node[block, right of=data-clean] (model-training) {Entrenamiento del modelo};
    \node[block, right of=model-training] (model-testing) {Evaluación del modelo};
    \node[block, right of=model-testing] (prediction) {Predicción};

\draw[line] (estudiante) -- (data-ingestion)
            (data-ingestion) -- (data-clean)
            (data-clean) -- (model-training)
            (model-training) -- (model-testing)
            (model-testing) -- (prediction);

\path[line,dashed] 
   (model-testing.south) -- +(0,-.35) -| (model-training.south)
   (prediction.south) -- +(0,-.60) -| (data-ingestion.south);

\end{tikzpicture}
%}}
	%Este funciona
	%%\centering\makebox[\textwidth]{
\begin{tikzpicture}[node distance=3cm, auto]

    % Place nodes
    \node[label={Estudiante},charlie,monitor,minimum size=1.3cm] (estudiante);
    \node[block, right of=estudiante] (data-ingestion) {Entrada de datos};
    \node[block, right of=data-ingestion] (data-clean) {Transformación de datos};
    \node[block, right of=data-clean] (model-training) {Entrenamiento del modelo};
    \node[block, right of=model-training] (model-testing) {Evaluación del modelo};
    \node[block, right of=model-testing] (prediction) {Predicción};

\draw[line] (estudiante) -- (data-ingestion)
            (data-ingestion) -- (data-clean)
            (data-clean) -- (model-training)
            (model-training) -- (model-testing)
            (model-testing) -- (prediction);

\path[line,dashed] 
   (model-testing.south) -- +(0,-.35) -| (model-training.south)
   (prediction.south) -- +(0,-.60) -| (data-ingestion.south);

\end{tikzpicture}
%}
	%\input{03_GraphicFiles/p06.pdf}
	\includegraphics[width=0.7\textwidth]{03_GraphicFiles/p06.png}
	\captionsource{Ciclo de construcción y perfecionamiento del modelo}{\fuentePropia}
	\label{fig:ml-pipeline}
\end{figure}


