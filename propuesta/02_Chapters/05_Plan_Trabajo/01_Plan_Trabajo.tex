\chapter{Plan de trabajo}
\label{ch:plan_trabajo}

El presente proyecto contempla 612 horas de trabajo efectivas y se realizará en el transcurso del segundo semestre del año 2017, el cual se inicia el 7 de agosto y termina el 7 de diciembre del presente año contemplando 16 semanas de trabajo. Se dispone como día de trabajo todos los días hábiles de la semana, y un horario de trabajo desde las 9:00 hasta las 18:00 hrs considerando una hora de descanso.

El plan de trabajo propuesto se muestra en la \tabla{tab:plan}, dada las metodologías empleadas las actividades se realizan de forma secuencial. Cabe destacar que a la fecha de entrega de este informe, el alumno candidato a tesista ya ha avanzado el estado del arte e investigación de tecnologías.

\begin{table}[htb]
  \centering
    \captiontable{Plan de trabajo propuesto}{\fuentePropia}
    \pgfplotstabletypeset[
        col sep=semicolon,
        columns/Actividad/.style={string type, column type=|l},
        columns/Duración (HH)/.style={string type},
        display columns/0/.style={column type={p{.7\textwidth}}},
        every even row/.style={before row={\rowcolor[gray]{0.9}}},
        every head row/.style={before row=\toprule,after row=\midrule},
        every last row/.style={after row=\bottomrule}
    ]{04_Tables/plan_trabajo.csv}
    \label{tab:plan}
\end{table}

%\begin{figure}[htb]
%    \centering
%    \scalebox{0.6}{\begin{ganttchart}[
    %today=4
    ]{1}{17}

    \gantttitle{Proyecto tesis}{17}\\
    \gantttitle[]{2017}{17} \\               
    
    \gantttitle{Ago}{4}                     
    \gantttitle{Sep}{4}
    \gantttitle{Oct}{4}
    \gantttitle{Nov}{4}
    \gantttitle{Dic}{1} \\ % Fin: Entrega 7 diciembre 2017
    
    \ganttgroup[inline=false]{Actualización del estado del arte centrado en trabajos recientes}{1}{8} \\ 
    \ganttgroup[inline=false]{Exploración, preprocesamiento y transformación de los registros del proceso de búsqueda de información obtenidos por NEURONE}{1}{17} \\
    \ganttgroup[inline=false]{Definición de las características para la construcción de modelos de predicción del desempeño de búsqueda de estudiantes}{1}{17} \\ 
    \ganttgroup[inline=false]{Comparación y selección de los algoritmos y/o técnicas de minería de datos para la construcción de modelos}{1}{17} \\ 
    \ganttgroup[inline=false]{Validación de los algoritmos y/o técnicas con datos conocidos}{1}{17} \\ 
	\ganttgroup[inline=false]{Evaluación de los modelos construidos utilizando métricas de desempeño}{1}{17} \\ 

    \ganttgroup[progress=70, inline=false]{Redacción del documento}{1}{17} \\ 
\end{ganttchart}}
%    %\begin{ganttchart}[
    %today=4
    ]{1}{17}

    \gantttitle{Proyecto tesis}{17}\\
    \gantttitle[]{2017}{17} \\               
    
    \gantttitle{Ago}{4}                     
    \gantttitle{Sep}{4}
    \gantttitle{Oct}{4}
    \gantttitle{Nov}{4}
    \gantttitle{Dic}{1} \\ % Fin: Entrega 7 diciembre 2017
    
    \ganttgroup[inline=false]{Actualización del estado del arte centrado en trabajos recientes}{1}{8} \\ 
    \ganttgroup[inline=false]{Exploración, preprocesamiento y transformación de los registros del proceso de búsqueda de información obtenidos por NEURONE}{1}{17} \\
    \ganttgroup[inline=false]{Definición de las características para la construcción de modelos de predicción del desempeño de búsqueda de estudiantes}{1}{17} \\ 
    \ganttgroup[inline=false]{Comparación y selección de los algoritmos y/o técnicas de minería de datos para la construcción de modelos}{1}{17} \\ 
    \ganttgroup[inline=false]{Validación de los algoritmos y/o técnicas con datos conocidos}{1}{17} \\ 
	\ganttgroup[inline=false]{Evaluación de los modelos construidos utilizando métricas de desempeño}{1}{17} \\ 

    \ganttgroup[progress=70, inline=false]{Redacción del documento}{1}{17} \\ 
\end{ganttchart}
%    \captionsource{Carta Gantt propuesta.}{\fuentePropia}
%    \label{fig:gantt}
%\end{figure}