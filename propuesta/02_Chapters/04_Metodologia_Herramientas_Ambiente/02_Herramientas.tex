\section{Herramientas de desarrollo}
\label{sec:herramientas}
Las herramientas a utilizar en el trabajo de tesis, se dividen tanto en \ingles{hardware} como en \ingles{software}.

\subsection{Herramientas de \ingles{hardware}}
El desarrollo se llevará a cabo con procesador Intel Core i7 7ma Generación \textit{Kabylake} de 3.6 Ghz, con memoria Ram de 16 GB y 2 TB de disco duro. Además, los despliegues de prueba se realizan sobre un servidor privado virtual (VPS, por sus siglas en inglés) con el sistema operativo GNU/Linux Ubuntu Server alojado en el proveedor DigitalOcean\footnote{https://www.digitalocean.com/}.

\subsection{Herramientas de \ingles{software}}
En cuanto herramientas \ingles{software}, el desarrollo se llevará a cabo en la distribución GNU/Linux Debian en su versión 9.0. El modelo se llevará a cabo en Tensorflow. Para el análisis estadístico se hará uso de R. Además, cada módulo desarrollado estará contenido en contenedores de Docker para facilitar el despliegue en producción del modelo desarrollado. Todo el trabajo realizado, tanto código como documento escrito estará bajo el sistema de control de versiones Git. Finalmente, se hará uso de \LaTeX\ para el documento escrito.