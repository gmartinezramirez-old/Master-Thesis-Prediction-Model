\section{Revisión de la literatura}
\label{sec:revision-literatura}
En esta sección se presenta el estado del arte que da soporte a este trabajo, el cual comprende en primer lugar el estudio del comportamiento de estudiantes. En segundo lugar, técnicas de minería de datos y plataformas de aprendizaje de máquina aplicada al contexto educacional.

%\subsection{Plataformas de aprendizaje de máquinas aplicadas al contexto educacional}

Cuando se aplica minería de datos en instituciones educativas, la disciplina se conoce como minería de datos educacional (MDE, desde ahora en adelante).

La MDE es una disciplina en evolución que usa tecnologías informáticas, como son almacenes de datos y herramientas de inteligencia de negocios para descubrir tendencias y patrones sobre datos educacionales. El conocimiento que la MDE genera apoya a las autoridades de centros de educación en la toma de decisiones oportunas, y a los profesores para analizar el comportamiento y aprendizaje de sus alumnos \parencite{romero2010educational}. La disciplina se enfoca en el diseño de modelos para mejorar las experiencias del aprendizaje y la eficiencia organizacional \parencite{pandey2013decision}. El principal objetivo de la MDE es visto por diferentes investigadores como: i) modelado del estudiante, ii) modelado del dominio, iii) sistema de aprendizaje, iv) construir modelos computacionales, y v) estudiar los efectos de los recursos [7-9].

Actualmente, la aplicación de MDE se radica en universidades, tales como, Paul Smith’s College, la cual utiliza sus datos históricos para mejorar las tasas de retención de alumnos \parencite{bichsel2012analytics}. En este contexto, University of Georgia desarrolló un modelo para predecir la tasa de graduación y abandono estudiantil en un ambiente en línea \parencite{morris2005predicting}. Finalmente, la Purdue University han usado MD para determinar que la evaluación en etapas tempranas y de forma frecuente permite cambiar los hábitos de los estudiantes con calificaciones bajo la media en cursos introductorios. El equipo de investigación ha desarrollado un sistema de alerta académica temprana para saber el desempeño de los estudiantes \cite{baepler2010academic}. 

%TODO: Expandir esto
\cite{baker2010data} desarrolló un modelo de predicción usando datos recopilados automáticamente de interacciones entre estudiantes y el \ingles{software} como variables de predicción, y después validando la precisión del modelo al ser generalizado a más estudiantes y contextos. Entonces fueron capaces de estudiar sus avances en el conjunto completo de datos.

%TODO: ¿Que es un buen modelo?
\cite{koedinger2015data} define que un buen modelo cognitivo de un estudiante debe ser capaz de predecir las diferencias en la dificultad de una tarea, y como el aprendizaje es transferido de tarea en tarea.

%TODO: ¿En base a que criterio son relevantes?
Las contribuciones de \cite{romero2010educational} son las relevantes en este campo hasta la fecha. Acercan la minería de datos al contexto educativo y describe los diferentes grupos de usuarios, tipos de entornos escolares y los datos que proporcionan. Luego, exponen las tareas más típicas en el ambiente escolar que pueden resueltas a través de técnicas de minería de datos.

\cite{sarala2015empirical} discute las aplicaciones de la minería de datos en instituciones educativas, para extraer la información útil de grandes conjuntos de datos (\ingles{datasets}), y proporciona herramientas analíticas para ver y utilizar esta información para tomar decisiones basadas en ejemplos de la vida real.

%TODO: Profundizar
\cite{dutt2015clustering} consolida las variantes de algoritmos de clustering aplicados al contexto de MDE. Además, simplifica el diseño los sistemas que aprenden de los datos, utilizando técnicas y algoritmos de minería de datos, tales como, clustering, clasificación y predicción.  

\cite{merceron2005educational} establece cómo los algoritmos de minería de datos pueden escoger información pedagógica importante. El conocimiento obtenido ayuda a mejorar el cómo administrar la clase, como el alumno aprende, y cómo proporcionar un feedback a los alumnos. 

\cite{akinola2012data} aplica técnicas de minería de datos aplicados a estudiar el rendimiento de estudiantes de educación universitaria en cursos de programación. Los resultados demuestran que el conocimiento a priori de física y matemáticas influye de forma positiva en el rendimiento en la programación. En este mismo contexto, \cite{lahtinen2005study} estudia las dificultades de aprender programación, con el objetivo de crear material adecuado para introducir el curso a los estudiantes. De este estudio, se obtuvo las dificultades que sufren los estudiantes al momento de enfrentar tareas de programación.

%TODO: FIX THIS
\cite{borkar2014attributes} evalúa el rendimiento de los estudiantes, donde selecciona algunos atributos mediante minería de datos, haciendo uso de una red neuronal multicapa perceptrón y usando una validación cruzada selecciona las características más influyentes, estableciendo las reglas necesarias para poder detectar las características necesarias para poder predecir el rendimiento de los estudiantes. \cite{jayakameswaraiah2014study} aplica los mismos métodos propuestos por \cite{borkar2014attributes}.

\cite{abdullah2014students} realiza un sistema de predicción del rendimiento de los estudiantes basado en la actividad actual, y mediciones anteriores, clasificando cuales estudiantes rendirán bien, y los que no. 

\cite{oskouei2014predicting} identifica los factores que afectan el rendimiento de los estudiantes en diferentes países, y aplica técnicas de clasificación y predicción para mejorar la precisión de las predicciones de los resultados de los estudiantes. Los resultados muestran que los factores de género, entorno familiar, nivel de educación de los padres, y el estilo de vida, afectan el rendimiento académico de los estudiantes, independiente del país.

En esta misma línea, \cite{borkar2013predicting} sugiere un método de evaluación del rendimiento de los estudiantes, usando reglas asociativas de minería de datos, estimando el resultado de los estudiantes basado en la asistencia a sus cursos y su avance académico. \cite{shazmeen2013performance} evalúa el rendimiento de diferentes algoritmos de clasificación y análisis predictivo, basado en el trabajo de \cite{borkar2013predicting} y propone técnicas de preprocesamiento de datos para lograr mejores resultados. 

Tal como se muestra en los antecedentes anteriores, las investigaciones en MDE se realizan mayoritariamente en aprendizaje \ingles{online} y en casos puntuales en educación superior, por lo que es limitada la información respecto a educación básica o media, específicamente en la predicción de errores y fracaso escolar \parencite{marquez2013predicting}. Para mayor información de trabajos relacionados con la MDE, consultar los siguientes reviews [30-33]. 