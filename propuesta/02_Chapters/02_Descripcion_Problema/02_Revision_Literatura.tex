\section{Revisión de la literatura}
\label{sec:revision-literatura}
Esta sección tiene como objetivo entregar las bases teóricas, conceptuales y empíricas que soportan el desarrollo de esta investigación. En primer lugar, se presenta el marco conceptual donde se entregan las definiciones y conceptos necesarios para abordar esta investigación. Finalmente, se presenta el estado del arte relacionado con el tema.

\subsection{Marco conceptual}

\subsection*{Alfabetización informacional}
La alfabetización informacional (conocida en ingles \ingles{information literacy}) es definida como “el grupo de habilidades en las que se requiere reconocer cuándo la información es necesaria y tener la habilidad de encontrar, evaluar y usar efectivamente dicha información necesaria” \footnote{\traduccionlibre} \parencite[p.~2]{american2000information}. Es un campo que cubre varias áreas, entre las que se destaca la alfabetización digital, las habilidades de uso de bibliotecas, la ética informacional, la lectura crítica, el pensamiento crítico, los derechos de autor, la seguridad y privacidad, entre otras. A través del estudio de estas áreas como factores que influyen a la alfabetización informacional se puede obtener una visión clara de cómo los estudiantes llevan a cabo sus tareas de obtención y selección de información.

\subsection*{Competencias de investigación en línea}
Se definen las competencias de investigación (\ingles{inquiry skills} en inglés) como “las habilidades para explorar preguntas, para poder reunir, interpretar y sintetizar diferentes tipos de información y datos, además de desarrollar y compartir una explicación para responder preguntas dadas” \footnote{\traduccionlibre} \parencite[p.~13]{national2000inquiry}. En base a este concepto nacen las competencias de investigación en línea (conocidas en inglés como \ingles{online inquiry skills}), que son una instancia específica de las competencias de investigación, pero aplicada sobre información disponible en línea \parencite{quintana2005framework}.

Las competencias de investigación en línea involucran una serie de actividades cognitivas, como generar una pregunta de investigación, buscar información relevante en colecciones digitales, evaluar y seleccionar la información encontrada, e integrar coherentemente la información seleccionada para responder la pregunta original \parencite{eisenberg1990information}.

\subsection*{Minería de datos educacional}
La minería de datos utiliza una combinación de bases de conocimientos explícita, conocimientos analíticos complejos y conocimiento de campo para descubrir las tendencias y los patrones ocultos, estas tendencias y patrones forman la base de los modelos predictivos que permiten a los analistas realizar nuevas observaciones de los datos existentes \parencite{luan2002data}. La gran cantidad de información generada hoy en día por los estudiantes permite que la minería de datos obtenga datos relevantes y, a través de métodos estadísticos y otras herramientas, relacione la información para conocer si el proceso de enseñanza aprendizaje ha dado resultados positivos. 

\textcite[p.~9]{mining2012enhancing} define la minería de datos educacional (MDE, desde ahora en adelante) como “la teoría que desarrolla métodos, aplica técnicas estadísticas y de aprendizaje automático para analizar los datos recogidos durante el proceso de la enseñanza y aprendizaje” \footnote{\traduccionlibre}. Actualmente, los usos más generales que se le están dando a la MDE básicamente se enfocan en mejorar la estructura del conocimiento y determinar el apoyo pedagógico al estudiante.

\subsection{Estado del arte}

\subsection*{Usos de la minería de datos educacional}

%TODO: Expandir esto
%\textcite{baker2010data} desarrolló un modelo de predicción usando datos recopilados automáticamente de interacciones entre estudiantes y el \ingles{software} como variables de predicción, y después validando la precisión del modelo al ser generalizado a más estudiantes y contextos. Entonces fueron capaces de estudiar sus avances en el conjunto completo de datos.

Actualmente, la aplicación de la MDE radica en universidades, tales como Paul Smith’s College, la cual utiliza sus datos históricos para mejorar las tasas de retención de alumnos \parencite{bichsel2012analytics}. En este contexto, University of Georgia desarrolló un modelo para predecir la tasa de graduación y abandono estudiantil, el cual se alimenta en base la información recopilada \parencite{morris2005predicting}. Finalmente, la Purdue University han usado MDE para determinar que la evaluación en etapas tempranas y de forma frecuente permite cambiar los hábitos de los estudiantes con calificaciones bajo la media en cursos introductorios, en base a este trabajo, el mismo equipo de investigación desarrollo un sistema de alerta académica temprana para saber el desempeño de los estudiantes \parencite{baepler2010academic}. 

\textcite{merceron2005educational} establece cómo los algoritmos de minería de datos pueden escoger información pedagógica importante. El conocimiento obtenido ayuda a mejorar el cómo administrar la clase, como el alumno aprende, y cómo proporcionar un feedback a los alumnos. Basado en este trabajo, \textcite{abdullah2014students} realiza un sistema de predicción del rendimiento de los estudiantes basado en la actividad actual y mediciones anteriores clasificando cuales estudiantes rendirán bien y los que no. 

\textcite{henrie2015measuring} clasifica los datos generados por los estudiantes en un sistema computacional en tres categorías: comportamiento, cognitivas y emocionales. El comportamiento de los estudiantes es una de las más estudiadas, en esta categoría se estudian las variables cuantitativas: resultados de consultas realizadas en un motor de búsqueda, teclas presionadas, rastreo ocular, tareas de búsqueda, esfuerzo (intentos por finalizar tareas asignadas), participación, tiempos de permanencia o de respuestas y uso de sitios \ingles{web}, entre otros. Tomando tiempos de permanencia y el uso de sitios, \textcite{Shah2016} presenta diversas métricas para evaluar el rendimiento de la búsqueda de información, en base a las distintas acciones hechas por usuarios. Tales métricas se usan con el objetivo de pronosticar la probabilidad de un usuario de tener éxito en el futuro, en base a su desempeño actual. 


\subsection*{Técnicas utilizadas en la minería de datos educacional}


\textcite{chen2008integrated} evalúa el rendimiento académico de estudiantes de pregrado estudiando datos académicos del Departamento de Ciencias de la Computación de National Defence University of Malaysia (NUDM) utilizando una combinación de técnicas de minería de datos, como ANN (\ingles{Artificial Neural Network}) y árboles de decisión como un método de clasificación con el que se producen ocho reglas para la identificación automática de los estilos cognitivos de los estudiantes basados en sus patrones de aprendizaje. Los hallazgos obtenidos se aplicaron para desarrollar un modelo que pueda apoyar el desarrollo de programas educativos \ingles{web}.

\textcite{moreno2009data} predice la probabilidad de que los estudiantes de acuerdo a sus registros académicos históricos fallen en un curso \ingles{online} en Moodle\footnote{https://moodle.org/} haciendo uso de las técnicas de maximización de la entropía, el método de agrupamiento K-means y X-means, usando el \ingles{software} WEKA.

%TODO: Profundizar
%Dutt es un review
%\textcite{dutt2015clustering} consolida las variantes de algoritmos de clustering aplicados al contexto de MDE. Además, simplifica el diseño los sistemas que aprenden de los datos, utilizando técnicas y algoritmos de minería de datos, tales como, clustering, clasificación y predicción.

\textcite{lahtinen2005study} estudia las dificultades de aprender programación con el objetivo de crear material adecuado para introducir el curso a los estudiantes utilizando el método de agrupamiento K-means y \ingles{Hierarchical clustering} (más conocido como Ward's \ingles{clustering}, a través de este estudio se obtuvo las dificultades que sufren los estudiantes al momento de enfrentar tareas de programación. Basado en este trabajo \textcite{akinola2012data} aplica ANN para predecir el resultado de los cursos de programación en estudiantes de pregrado basados en su historial académico, los resultados de este estudio muestran que los estudiantes con un conocimiento a priori de física y matemática tienen mejor desempeño en los cursos que el resto.

\textcite{borkar2014attributes} evalúa el rendimiento de los estudiantes, donde selecciona algunos atributos mediante minería de datos, haciendo uso de una red neuronal multicapa perceptrón y usando una validación cruzada selecciona las características más influyentes, estableciendo las reglas necesarias para poder detectar las características necesarias para poder predecir el rendimiento de los estudiantes. Basado en los métodos propuestos y el mismo conjunto de datos de este trabajo, \textcite{jayakameswaraiah2014study} compara los métodos de perceptrón multicapa, Naive Bayes, SMO y J48 con el objetivo de obtener el mejor algoritmo de clasificación y predicción entre todos ellos. De todos los métodos comparados, el método de perceptrón multicapa obtuvo un \ingles{accuracy} de 75\%.


\subsection*{Identificación de factores}
\textcite{borkar2013predicting} sugiere un método de evaluación del rendimiento de los estudiantes usando reglas asociativas de minería de datos, estimando el resultado de los estudiantes basado en la asistencia a sus cursos y su avance académico. Basado en este trabajo, \textcite{shazmeen2013performance} evalúa el rendimiento de diferentes algoritmos de clasificación y análisis predictivo, proponiendo técnicas de preprocesamiento de datos para lograr mejores resultados.

\textcite{oskouei2014predicting} identifica que los factores que afectan el rendimiento de los estudiantes de primer semestre de la carrera de Ingeniería de Software de Irán e India, aplicando técnicas de clasificación y predicción para mejorar la precisión de las predicciones de los resultados de los estudiantes. Los resultados muestran que los factores de género, entorno familiar, nivel de educación de los padres, y el estilo de vida afectan el rendimiento académico de los estudiantes independiente del país.

Tal como se muestra en los antecedentes anteriores, las investigaciones en MDE se realizan mayoritariamente en aprendizaje \ingles{online} y en casos puntuales en educación superior, por lo que es limitada la información respecto a educación básica o media, específicamente en la predicción de errores y fracaso escolar. Para mayor información de trabajos relacionados con la MDE, consultar los siguientes \ingles{reviews} \parencite{shahiri2015review,sukhija2015recent,anoopkumar2016review,dutt2017systematic}.