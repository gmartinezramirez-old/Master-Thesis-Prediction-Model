\section{Definición del problema}
\label{sec:definicion_problema}
El proceso de búsqueda de información involucra diferentes procesos cognitivos, habilidades, variables de comportamiento y entorno de una persona. En el contexto de la enseñanza de la alfabetización informacional, las evaluaciones de los cursos se centran principalmente en los resultados de los estudiantes, sin tomar en cuenta el proceso formativo y factores asociados que podrían influir directa o indirectamente sobre los resultados finales y el desempeño de los alumnos.  

A partir de lo señalado anteriormente, surgen las siguientes interrogantes (\ingles{research questions}, RQ desde ahora en adelante):

\begin{description}
	\item [RQ 1]: ¿De qué manera se puede estimar, durante el proceso de aprendizaje de competencias informacionales, la influencia de diversos factores en el desempeño de los estudiantes?
	\item [RQ 2]: ¿En qué medida es posible detectar situaciones anormales de conducta, y determinar las causas que llevan a un estudiante fallar durante el proceso de búsqueda de información? 
	%TODO: Preguntas centradas en la implementación tecnologica de estos modelos. Ej: NEURONE
	\item [RQ 3]:
\end{description}