\chapter{Descripción del problema}
\label{chp:descripcion-problema}


\section{Motivación}
\label{sec:motivacion}

\section{Revisión de la literatura}
\label{sec:revision-literatura}

\section{Definición del problema}
\label{sec:definicion_problema}
El proceso de búsqueda de información involucra diferentes procesos cognitivos, habilidades, variables de comportamiento y entorno de una persona. En el contexto de la enseñanza de la alfabetización informacional, las evaluaciones de los cursos se centran principalmente en los resultados de los estudiantes, sin tomar en cuenta el proceso formativo y factores asociados que podrían influir directa o indirectamente sobre los resultados finales y el desempeño de los alumnos.  

A partir de lo señalado anteriormente, surgen las siguientes interrogantes (\ingles{research questions}, RQ desde ahora en adelante):

\begin{description}
	\item [RQ 1]: 
	\item [RQ 2]:
\end{description}