\section{Motivación}
\label{sec:motivacion}
La alfabetización informacional (conocida en inglés como \ingles{information literacy}) es definida como “el grupo de habilidades en las que se requiere reconocer cuándo la información es necesaria y tener la habilidad de encontrar, evaluar y usar efectivamente dicha información necesaria”\footnote{\traduccionlibre} \parencite[p.~2]{american2000information}. Durante la última década, debido a los rápidos avances de las tecnologías de la información y comunicación (TICs) ha aumentado la cantidad de recursos digitales en Internet y además se ha facilitado el acceso a ellos. Estos avances han provocado una brecha entre el ser humano y la habilidad de reconocer cuando la información es necesaria para satisfacer su necesidad de búsqueda, la cual se puede asociar principalmente a dos razones: En primer lugar, las competencias de alfabetización informacional no son enseñadas ni reforzadas a temprana edad. Segundo, las búsquedas \ingles{web} han pasado a ser parte de las tareas comunes que realizan los estudiantes, disminuyendo las visitas a bibliotecas y el uso de fuentes revisadas.

Considerando la diversidad de fuentes de información y tipos de recursos en línea, resulta necesario desarrollar competencias informacionales durante el proceso de formación en los distintos niveles educativos (básica, media y universitaria). La enseñanza de la alfabetización informacional se imparte principalmente por bibliotecas universitarias, y en menor medida en la etapa escolar obligatoria \parencite{weiner2014teaches}. En Chile, la enseñanza de competencias informacionales es cubierta en bibliotecas universitarias y cursos introductorios de mallas universitarias \parencite{marzal2015diagnostico}. De acuerdo con \textcite{urra2016alfabetizacion}, los estudiantes universitarios de Chile presentan problemas con las competencias informacionales, ya que no aplican la búsqueda de información de forma crítica ni sistemática. Una de las posibles causas de por qué los estudiantes tienen dificultades con estas competencias es el hecho de que en los colegios y en el inicio de su educación se prioriza la reiteración de la información. Las consecuencias de no considerar cuándo y por qué se necesita la información, dónde encontrarla y cómo evaluarla, se ven reflejadas en la evaluación crítica de la información, y en el desempeño de los estudiantes \parencite{urra2016alfabetizacion}. 

\textcite[p.~475]{head2013project} a través de encuestas a estudiantes universitarios, establece que al momento de realizar investigaciones el 84\% de los estudiantes universitarios utiliza como fuente primaria de búsqueda Wikipedia\footnote{https://es.wikipedia.org/} y un 87\% consulta a sus amigos, sin verificar la veracidad de la información que obtienen. Como consecuencia, los estudiantes al no ser instruidos en parafrasear, resumir o citar fuentes revisadas, caen al plagio de forma premeditada o no intencionada. 

% Thesis Restatement
A partir de los argumentos anteriormente expuestos, respecto a la enseñanza de competencias de alfabetización informacional se puede observar que no ha sido completamente satisfecha y la brecha entre los usuarios e alfabetización informacional permanece abierta.

Esta propuesta de tesis se enmarca en el contexto del proyecto de investigación “\ingles{Enhancing Learning and Teaching Future Competences of Online Inquiry in Multiple Domains}”\footnote{https://www.researchgate.net/project/Enhancing-learning-and-teaching-for-future-competences-of-online-inquiry-in-multiple-domains-iFuCo} (iFuCo, desde ahora en adelante), el cual pretende abordar la temática de la alfabetización informacional en estudiantes de enseñanza básica (5to y 6to básico) con el objetivo de estudiar sus patrones de comportamiento y ofrecer mallas curriculares y asignaturas adecuadas respecto al tema \parencite{sormen2017performance}.