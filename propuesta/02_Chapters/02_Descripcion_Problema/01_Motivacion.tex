\section{Motivación}
\label{sec:motivacion}
%Revisar estilo OREO
% O: your opinion. Tell me what you think
% R: what are your reasons?
% E: expalin through examples
% O: restate your opinion

La alfabetización informacional (conocida en ingles \ingles{information literacy}) es definida como “el grupo de habilidades en las que se requiere reconocer cuándo la información es necesaria y tener la habilidad de encontrar, evaluar y usar efectivamente dicha información necesaria” \footnote{Traducción libre} \parencite[p.~2]{american2000information}. Es un campo que cubre varias áreas, entre las que se destaca la alfabetización digital, las habilidades de uso de bibliotecas, la ética informacional, la lectura crítica, el pensamiento crítico, los derechos de autor, la seguridad y privacidad, entre otras. A través del estudio de estas áreas como factores que influyen a la alfabetización informacional se puede obtener una visión clara de cómo los estudiantes llevan a cabo sus tareas de obtención y selección de información.

%Se definen las competencias de investigación (\ingles{inquiry skills} en inglés) como “las habilidades para explorar preguntas, para poder reunir, interpretar y sintetizar diferentes tipos de información y datos, además de desarrollar y compartir una explicación para responder preguntas dadas” \footnote{Traducción libre} \parencite[p.~13]{national2000inquiry}. En base a este concepto nacen las competencias de investigación en línea (conocidas en inglés como \ingles{online inquiry skills}), que son una instancia específica de las competencias de investigación, pero aplicada sobre información disponible en línea \parencite{quintana2005framework}. Las competencias de investigación en línea involucran una serie de actividades cognitivas, como generar una pregunta de investigación, buscar información relevante en colecciones digitales, evaluar y seleccionar la información encontrada, e integrar coherentemente la información seleccionada para responder la pregunta original \parencite{eisenberg1990information}.

% %Intro/Definición/Gancho
Durante la última década, debido a los rápidos avances de las tecnologías de la información y comunicación (TICs, desde ahora en adelante) ha aumentado la cantidad de recursos digitales en Internet y la diversidad de fuentes de información. Además, se ha facilitado el acceso a estos. Asimismo, las búsquedas \ingles{web} han pasado a ser parte de las tareas comunes que realizan los estudiantes de los planteles educativos. En consecuencia, se ha disminuido las visitas a bibliotecas, y el uso de fuentes revisadas y editadas.

%La alfabetización en información es una disciplina que se define a sí misma en base al desarrollo de destrezas, habilidades y competencias informacionales que permitan ir fortaleciendo el aprendizaje constante y el trabajo colaborativo \parencite{american2000information}. Además, favorece la capacidad de buscar, clasificar, y comprender la información, para posteriormente convertirla en conocimiento asimilado y útil. A causa de esto, el estudio, análisis y modelado de las conductas de los estudiantes en ambientes de búsqueda \ingles{web} es esencial para comprender sus niveles de alfabetización informacional \parencite{tseng2009meta}.

Actualmente, en Chile la enseñanza de competencias informacionales es cubierta en bibliotecas universitarias y cursos introductorios de mallas universitarias \parencite{marzal2015diagnostico}. De acuerdo con \textcite{urra2016alfabetizacion}, los estudiantes universitarios de Chile presentan problemas con las competencias informacionales, ya que no aplican la búsqueda de información de forma crítica. Una de las posibles causas de por qué los estudiantes tienen dificultades con estas competencias es el hecho de que en los colegios y en el inicio de su educación se prioriza la reiteración de la información.

Las consecuencias de no considerar cuándo y por qué se necesita la información, dónde encontrarla, y cómo evaluarla, se ven reflejadas en la evaluación crítica de la información, y en el desempeño de los estudiantes \parencite{urra2016alfabetizacion}. A causa de esto, existe la necesidad de estudiar el fenómeno de la alfabetización informacional y las competencias de investigación en línea con los objetivos de i) conocer y estudiar los comportamientos de los estudiantes en tareas de búsqueda de información en medios digitales, y ii) obtener modelos para reforzar los niveles de alfabetización informacional.

En base a los argumentos anteriormente expuestos, se plantea la necesidad de . Luego, se considera la

% Thesis Restatement
Esta propuesta de tesis se enmarca en el contexto del proyecto de investigación “\ingles{Enhancing Learning and Teaching Future Competences of Online Inquiry in Multiple Domains}” (iFuCo, desde ahora en adelante)\footnote{https://www.researchgate.net/project/Enhancing-learning-and-teaching-for-future-competences-of-online-inquiry-in-multiple-domains-iFuCo}, el cual pretende abordar la temática de la alfabetización informacional en estudiantes de enseñanza básica con el objetivo de estudiar sus patrones de comportamiento y ofrecer modelos curriculares adecuados respecto al tema \parencite{sormen2017performance}.
