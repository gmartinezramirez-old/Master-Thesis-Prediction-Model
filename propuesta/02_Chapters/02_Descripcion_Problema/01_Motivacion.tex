\section{Motivación}
\label{sec:motivacion}
La alfabetización informacional (conocida en inglés como \ingles{information literacy}) es definida como “el grupo de habilidades en las que se requiere reconocer cuándo la información es necesaria y tener la habilidad de encontrar, evaluar y usar efectivamente dicha información necesaria”\footnote{\traduccionlibre} \parencite[p.~2]{american2000information}. Durante la última década, debido a los rápidos avances de las tecnologías de la información y comunicación (TICs) ha aumentado la cantidad de recursos digitales en Internet y además se ha facilitado el acceso a ellos. Estos avances han provocado una brecha entre el ser humano y la habilidad de reconocer cuando la información es necesaria para satisfacer su necesidad de búsqueda, la cual se puede asociar principalmente a dos razones: En primer lugar, las competencias de alfabetización informacional no son enseñadas ni reforzadas a temprana edad. Segundo, las búsquedas \ingles{web} han pasado a ser parte de las tareas comunes que realizan los estudiantes, disminuyendo las visitas a bibliotecas y el uso de fuentes revisadas.

Considerando la diversidad de fuentes de información y tipos de recursos en línea, resulta necesario desarrollar competencias informacionales durante el proceso de formación en los distintos niveles educativos (básica, media y universitaria). La enseñanza de la alfabetización informacional se imparte principalmente por bibliotecas universitarias, y en menor medida en la etapa escolar obligatoria \parencite{weiner2014teaches}. En Chile, la enseñanza de competencias informacionales es cubierta en bibliotecas universitarias y cursos introductorios de mallas universitarias \parencite{marzal2015diagnostico}. De acuerdo con \textcite{urra2016alfabetizacion}, los estudiantes universitarios de Chile presentan problemas con las competencias informacionales, ya que no aplican la búsqueda de información de forma crítica. Una de las posibles causas de por qué los estudiantes tienen dificultades con estas competencias es el hecho de que en los colegios y en el inicio de su educación se prioriza la reiteración de la información. Las consecuencias de no considerar cuándo y por qué se necesita la información, dónde encontrarla y cómo evaluarla, se ven reflejadas en la evaluación crítica de la información, y en el desempeño de los estudiantes \parencite{urra2016alfabetizacion}. 

A través de encuestas a estudiantes universitarios, \textcite[p.~475]{head2013project} establece que al momento de realizar investigaciones el 84\% de los estudiantes universitarios utiliza como fuente primaria de búsqueda Wikipedia\footnote{https://es.wikipedia.org/} y un 87\% consulta a sus amigos, sin verificar la veracidad de la información que obtienen. Como consecuencia, los estudiantes al no ser instruidos en parafrasear, resumir o citar fuentes revisadas, caen al plagio de forma premeditada o no intencionada. 

% Thesis Restatement
A partir de los argumentos anteriormente expuestos, respecto a la enseñanza de competencias de alfabetización informacional, se puede ver que no ha sido completamente satisfecha y la brecha entre los usuarios e alfabetización informacional permanece abierta.

Esta propuesta de tesis se enmarca en el contexto del proyecto de investigación “\ingles{Enhancing Learning and Teaching Future Competences of Online Inquiry in Multiple Domains}”\footnote{https://www.researchgate.net/project/Enhancing-learning-and-teaching-for-future-competences-of-online-inquiry-in-multiple-domains-iFuCo} (iFuCo, desde ahora en adelante), el cual pretende abordar la temática de la alfabetización informacional en estudiantes de enseñanza básica con el objetivo de estudiar sus patrones de comportamiento y ofrecer modelos curriculares adecuados respecto al tema \parencite{sormen2017performance}.


%
%En primer lugar, debido a la diversidad de fuentes de información es necesario

%opcion 2
%Del mismo modo, las búsquedas \ingles{web} han pasado a ser parte de las tareas comunes que realizan estudiantes, disminuyendo las visitas a bibliotecas, y el uso de fuentes revisadas. Es en este contexto, que existe la necesidad de estudiar el fenómeno de la alfabetización informacional y las competencias de investigación en línea de los estudiantes con los objetivos de conocer, estudiar sus hábitos. En primer lugar, . En segundo lugar, . Finalmente, 

%existe la necesidad de 
%En primer lugar, . En segundo lugar, . Finalmente,

%Se definen las competencias de investigación (\ingles{inquiry skills} en inglés) como “las habilidades para explorar preguntas, para poder reunir, interpretar y sintetizar diferentes tipos de información y datos, además de desarrollar y compartir una explicación para responder preguntas dadas” \footnote{\traduccionlibre} \parencite[p.~13]{national2000inquiry}. En base a este concepto nacen las competencias de investigación en línea (conocidas en inglés como \ingles{online inquiry skills}), que son una instancia específica de las competencias de investigación, pero aplicada sobre información disponible en línea \parencite{quintana2005framework}. Las competencias de investigación en línea involucran una serie de actividades cognitivas, como generar una pregunta de investigación, buscar información relevante en colecciones digitales, evaluar y seleccionar la información encontrada, e integrar coherentemente la información seleccionada para responder la pregunta original \parencite{eisenberg1990information}.

% %Intro/Definición/Gancho
%Durante la última década, debido a los rápidos avances de las tecnologías de la información y comunicación (TICs, desde ahora en adelante) ha aumentado la cantidad de recursos digitales en Internet y la diversidad de fuentes de información. Además, se ha facilitado el acceso a estos. Asimismo, las búsquedas \ingles{web} han pasado a ser parte de las tareas comunes que realizan los estudiantes de los planteles educativos. En consecuencia, se ha disminuido las visitas a bibliotecas, y el uso de fuentes revisadas y editadas.

%La alfabetización en información es una disciplina que se define a sí misma en base al desarrollo de destrezas, habilidades y competencias informacionales que permitan ir fortaleciendo el aprendizaje constante y el trabajo colaborativo \parencite{american2000information}. Además, favorece la capacidad de buscar, clasificar, y comprender la información, para posteriormente convertirla en conocimiento asimilado y útil. A causa de esto, el estudio, análisis y modelado de las conductas de los estudiantes en ambientes de búsqueda \ingles{web} es esencial para comprender sus niveles de alfabetización informacional \parencite{tseng2009meta}.

%Estos avances han provocado una brecha entre el ser humano y la habilidad de reconocer cuando la información es necesaria para satisfacer su necesidad de búsqueda
%SP1: En primer lugar, las competencias de alfabetización informacional no son enseñadas ni reforzadas a temprana edad.

% Actualmente, ciertas universidades consideran las competencias informacionales como un requisito de entrada para iniciar estudios universitarios \parencite{smith2013information}

%SP2: Segundo, las búsquedas \ingles{web} han pasado a ser parte de las tareas comunes que realizan los estudiantes, disminuyendo las visitas a bibliotecas, y el uso de fuentes revisadas.


% Por ahora no
%El plagio es la causa más notoria de carencias en educación y valores en ambientes escolares, lo cual también se refleja en el hecho de que los estudiantes entreguen trabajos descargados de Internet, guiados solo por el deseo de obtener buenas calificaciones \parencite{adam2016student}.

%Consecuencias: Thesis restatement?


%A causa de esto, existe la necesidad de estudiar el fenómeno de la alfabetización informacional y las competencias de investigación en línea con los objetivos de i) conocer y estudiar los comportamientos de los estudiantes en tareas de búsqueda de información en medios digitales, y ii) obtener modelos para reforzar los niveles de alfabetización informacional.