%%% File encoding is UTF8
%%% You can use special characters just like ä,ü and ñ

% Chapter without numbering but with appearance in the Table of Contents
% \addchap is a command from KOMA-Script
%\addchap*{Abstract}

\chapter*{\centerline{Resumen}} \vspace{-6em}
El presente documento corresponde a la propuesta de tesis para la carrera de Ingeniería Civil en Informática y Magister en Ingeniería Informática, cuyo título es ``Modelo predictivo del desempeño de búsqueda de información en línea en estudiantes de educación básica''. A continuación se introduce el problema a resolver a lo largo del trabajo, las herramientas y metodologías  a emplear para abordar el problema.

Durante la última década, debido a los rápidos avances de las tecnologías de la información y comunicación ha aumentado la cantidad de recursos digitales en Internet, la diversidad de fuentes de información, y además, se ha facilitado el acceso a estos. Asimismo, las búsquedas \ingles{web} han pasado a ser parte de las tareas comunes que realizan los estudiantes de los planteles educativos. Considerando la diversidad de fuentes de información y tipos de recursos en línea, resulta necesario desarrollar competencias informacionales durante el proceso de formación en los distintos niveles educativos (básica, media y universitaria).

En el marco del proyecto iFuCo (\ingles{Enhancing learning and teaching future competences of online inquiry in multiple domains}), formado por investigadores de Chile y Finlandia, el cual desea investigar y modelar los comportamientos y competencias de investigación en línea de estudiantes de enseñanza básica, se propone la construcción de un modelo de predicción del desempeño de búsqueda de información en línea en estudiantes de educación básica el cual se vaya perfeccionando a través del registro de datos históricos y ofrezca retroalimentación tanto al estudiante como al docente. 

La investigación será guiada por la metodología Descubrimiento de Conocimiento en Base de Datos (conocido como KDD, las iniciales de \ingles{Knowledge Discovery in Databases}) con el fin de descubrir patrones en los datos que permitan la creación de un modelo de predicción del desempeño de búsqueda de información. Además, para apoyar el proceso de investigación, se desarrollará un módulo de la plataforma NEURONE (\ingles{oNlinE inqUiry expeRimentatiON systEm}). El módulo propuesto alimentará y perfeccionará el modelo de predicción y entregará predicciones de forma continua. El desarrollo de esta plataforma se guiará bajo la metodología Desarrollo de Rápido de Aplicaciones (conocido como RAD, las iniciales de \ingles{Rapid Application Development}) la cual se orienta a un desarrollo iterativo e incremental para la rápida construcción de prototipos de \ingles{software}.

\par\noindent
{\bfseries Palabras Claves\/}: Alfabetización informacional, Competencias informacionales, Estrategias de intervención.

%\newpage
%\chapter*{\centerline{Abstract}} \vspace{-6em}
%Today

%\par\noindent
%{\bfseries Keywords\/}: