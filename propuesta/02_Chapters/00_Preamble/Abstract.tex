%%% File encoding is UTF8
%%% You can use special characters just like ä,ü and ñ

% Chapter without numbering but with appearance in the Table of Contents
% \addchap is a command from KOMA-Script
%\addchap*{Abstract}

\chapter*{\centerline{Resumen}} \vspace{-6em}
Durante la última década, debido a los rápidos avances de las tecnologías de la información y comunicación ha aumentado la cantidad de recursos digitales en Internet, la diversidad de fuentes de información, y, además, se ha facilitado el acceso a estos. Asimismo, las búsquedas \ingles{web} han pasado a ser parte de las tareas comunes que realizan los estudiantes de los planteles educativos. Considerando la diversidad de fuentes de información y tipos de recursos en línea, resulta necesario desarrollar competencias informacionales durante el proceso de formación en los distintos niveles educativos (primaria, secundaria y universitaria).

En el marco del proyecto iFuCo (\ingles{Enhancing learning and teaching future competences of online inquiry in multiple domains}), formado por investigadores de Chile y Finlandia, el cual desea investigar y modelar los comportamientos y competencias de investigación en línea de estudiantes de enseñanza básica, se propone la construcción de un modelo de predicción del comportamiento de búsqueda de información en línea en estudiantes de educación básica el cual se vaya perfeccionando a través del registro de datos históricos y que de un feedback en tiempo real. 

La investigación será guiada por la metodología KDD con el fin de descubrir patrones en los datos que permitan la creación de un modelo de predicción del comportamiento de búsqueda. Además, para apoyar el proceso de investigación, se desarrollará una plataforma que funcione como extensión de la plataforma NEURONE (\ingles{oNlinE inqUiry expeRimentatiON systEm}). La plataforma propuesta alimentará y perfeccionará el modelo de predicción y entregará predicciones en tiempo real. Esta plataforma se guiará bajo la metodología RAD (\ingles{Rapid Application Development}) la cual se orienta a un desarrollo iterativo e incremental para la rápida construcción de prototipos de \ingles{software}.

%\par\noindent
%{\bfseries Palabras Claves\/}: Alfabetización informacional; Competencias informacionales; Estrategias de intervención.

%\newpage
%\chapter*{\centerline{Abstract}} \vspace{-6em}
%Today

%\par\noindent
%{\bfseries Keywords\/}: