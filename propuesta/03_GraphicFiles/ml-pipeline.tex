\begin{tikzpicture}
    \node[label={Estudiante},charlie,monitor,minimum size=0.9cm](estudiante);
    \node[block,right=of estudiante] (data) {Entrada de datos};
    \node[block,right=of data] (clean)  {Limpieza y transformación de datos};
    \node[block,right=of clean] (training)  {Entrenamiento del modelo};
    \node[block,right=of training] (testing)  {Prueba del modelo};
    \node[block,right=of testing] (prediction)  {Predicción};

 	% Draw edges
    \draw[arrow] (estudiante) -- (data);
    %\draw [arrow] (data) -- node [above] {text} (clean);
    \draw[arrow] (data) -- (clean);
    \draw[arrow] (clean) -- (training);
    \draw[arrow] (training) -- (testing);
    \draw[arrow] (testing) -- (prediction);
    %\draw[arrow] (testing.south) |-  (training.south);
    %\draw[arrow] (prediction.south) |-  (data.south);

    % NO 
    %\draw[arrow] (testing.south) -- +(-1,0) |- node[pos=0.25]  (training.south);
    % Fuente: https://tex.stackexchange.com/questions/102385/how-to-draw-a-return-arrow-from-node-3-to-node-1
    %\draw[arrow](testing.south) edge[du] (training.south)

\end{tikzpicture}

%testing a training: vertical-horizontal-vertical  -|-
%prediction a data: vertical-horizontal-vertical -|-


%Ejemplo 1: https://tex.stackexchange.com/questions/327138/tikz-matrix-arrow-from-empty-nodes-is-not-horizontal
%\begin{tikzpicture}
%  \matrix (m) [matrix of math nodes,row sep=3em,column sep=4em,
%               minimum width=2em]
%  {
%      c_0               & c_1 & c_2 & c_3 \\
%      {\phantom{c_0}}   & c_4 & c_5 & c_6 \\
%      {\phantom{c_0}}   & c_7 & c_8 &     \\
%  };
%
%  \path[-stealth]
%      (m-1-1) edge node [above] {$a_0$} (m-1-2)
%      (m-1-2) edge node [above] {$a_1$} (m-1-3)
%      (m-1-3) edge node [above] {$a_2$} (m-1-4)
%      (m-2-1) edge node [above] {$a_3$} (m-2-2)
%      (m-2-2) edge node [above] {$a_4$} (m-2-3)
%      (m-2-3) edge node [above] {$a_5$} (m-2-4)
%      (m-3-1) edge node [above] {$a_6$} (m-3-2)
%      (m-3-2) edge node [above] {$a_7$} (m-3-3);
%\end{tikzpicture}

%Ejemplo 2: https://tex.stackexchange.com/questions/168056/tikz-diagram-nodes-with-arrows
%\node[block] (a) {Ingesta de\\datos};
%\node[block] (b) {Limpieza\\ y transformación\\de datos};
%\node[block] (c) {Entrenamiento del modelo};
%\node[block] (d) {Prueba del modelo};
%\node[block] (e) {Predicción};
%\node[block, above right = 0.2cm and 2cm of a] (b) {Inter-Action \\Modeling};
%\node[block, below =2cm of b]   (c){Responsibilites\\ Modeling};
%\node[block, right =2cm of b]   (d){Interaction\\ Modeling};
%\node[block, right =2cm of c]   (e){Work in-group\\ Task Modeling};

%\draw[line] (a.north) |- (b.west);
%\draw[line] (a.south) |- (c.west);
%\draw[line] (e.north) -- (d.south);
%\draw[line] ([xshift=-1cm]b.south) -- ([xshift=-1cm]c.north);
%\draw[line] ([xshift=1cm]c.north) -- ([xshift=1cm]b.south);
%\draw[] (b.east) -- ++(10pt,0) coordinate[yshift=-1.7cm](l){} |- (c.east);
%\draw[<->,>=latex'] (d.west) -- ++(-10pt,0) coordinate[yshift=-1.7cm,](r){} |- (e.west);
%\draw[-] ([xshift=1cm]c.north) -- ([xshift=1cm]b.south);
%\draw[line] (l) -- (r);