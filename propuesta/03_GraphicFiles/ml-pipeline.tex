%\centering\makebox[\textwidth]{
\begin{tikzpicture}[node distance=3cm, auto]

    % Place nodes
    \node[label={Estudiante},charlie,monitor,minimum size=1.3cm] (estudiante);
    \node[block, right of=estudiante] (data-ingestion) {Entrada de datos};
    \node[block, right of=data-ingestion] (data-clean) {Transformación de datos};
    \node[block, right of=data-clean] (model-training) {Entrenamiento del modelo};
    \node[block, right of=model-training] (model-testing) {Evaluación del modelo};
    \node[block, right of=model-testing] (prediction) {Predicción};

\draw[line] (estudiante) -- (data-ingestion)
            (data-ingestion) -- (data-clean)
            (data-clean) -- (model-training)
            (model-training) -- (model-testing)
            (model-testing) -- (prediction);

\path[line,dashed] 
   (model-testing.south) -- +(0,-.35) -| (model-training.south)
   (prediction.south) -- +(0,-.60) -| (data-ingestion.south);

\coordinate[below=11mm of data-ingestion] (bottom brace);

\draw [my brace]
      (data-ingestion.south|-bottom brace) -- (model-training.south west|-bottom brace) 
      node[bottom label] {
        %Obtención de datos de NEURONE
        \begin{itemize}
        \item Spark
        \end{itemize}
      };

\draw [my brace]
      (model-training.south|-bottom brace) -- (prediction.south west|-bottom brace) 
      node[bottom label] {
        %Environmental \& Human factors
        \begin{itemize}
        \item Spark ML
        \end{itemize}
      };


\end{tikzpicture}
%}