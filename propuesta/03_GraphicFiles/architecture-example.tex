% tikz defintions
\usepackage{tikz}
\usetikzlibrary{fit}
\usetikzlibrary{arrows.meta}
\usetikzlibrary{positioning}
\usetikzlibrary{shapes.geometric}
\usetikzlibrary{decorations.pathreplacing}
\usetikzlibrary{calc} 
\usetikzlibrary{patterns}

\tikzstyle{roundBox} = [text width=6em, fill=white, minimum height=5em, rounded corners, text centered, draw=black]

\tikzset{
	arrow/.style={-latex, shorten >=0pt, shorten <=0pt},
	arrowboth/.style={<->/.tip = Latex, shorten >=0pt, shorten <=0pt}
}

\usepackage{xcolor}

\definecolor{bgcolor0}{RGB}{255, 231, 231} %red!10
\definecolor{strokecolor0}{RGB}{255, 181, 181} %red!30

\definecolor{bgcolor1}{RGB}{231, 231, 255} %blue!10
\definecolor{bgcolor2}{RGB}{255, 243, 231} %orange!10

\definecolor{unired}{RGB}{165,0,0}
\definecolor{bgcolor3}{RGB}{238, 205, 205} %unired!20

\tikzset{
	cmdarrow/.style={-latex, shorten >=0pt, shorten <=0pt, color = unired},
	eventarrow/.style={-latex, shorten >=0pt, shorten <=0pt, color = blue!60},
	queryarrow/.style={-latex, shorten >=0pt, shorten <=0pt, color = blue!60},
	cmdcircle/.style={circle, draw, minimum size=5, inner sep=0pt, fill=unired},
	querycircle/.style={circle, draw, minimum size=5, inner sep=0pt, fill=blue!60},
	eventcircle/.style={circle, draw, minimum size=5, inner sep=0pt, fill=blue!60}
}

\definecolor{cmdlabel}{RGB}{165, 0, 0} 
\definecolor{eventlabel}{RGB}{102, 102, 255}
\definecolor{querylabel}{RGB}{102, 102, 255} 

%\begin{document}

%-----#1 height, #2 width, #3 aspect, #4 name of the node, #5 coordinate, #6 label
\def\repositoryA[#1,#2,#3,#4,#5]#6{
	\node[
		draw, cylinder, alias=cyl, shape border rotate=90, aspect=#3, 
		minimum height=#1, minimum width=#2, outer sep=-0.5\pgflinewidth, 
		fill = white,
	] (#4) at #5 {};

	\node at #5 [text width=1.9cm, text centered] {#6};

	\fill [white] let \p1 = ($(cyl.before top)!0.5!(cyl.after top)$), 
		\p2 = (cyl.top), \p3 = (cyl.before top),
		\n1={veclen(\x3-\x1,\y3-\y1)},
		\n2={veclen(\x2-\x1,\y2-\y1)} in (\p1) ellipse (\n1 and \n2); 
}

% defines a box for the repositories (the storages)
%-----#1 height, #2 width, #3 aspect, #4 name of the node, #5 coordinate, #6 label
\def\repositoryB[#1,#2,#3,#4,#5]#6{
	\node[
		draw, cylinder, alias=cyl, shape border rotate=90, aspect=#3, 
		minimum height=#1, minimum width=#2, outer sep=-0.5\pgflinewidth, 
		fill = white
	] (#4) at #5 {};

	\node at #5 [text width=1.5cm, text centered] {#6};

	\fill [white] let \p1 = ($(cyl.before top)!0.5!(cyl.after top)$), 
		\p2 = (cyl.top), \p3 = (cyl.before top),
		\n1={veclen(\x3-\x1,\y3-\y1)},
		\n2={veclen(\x2-\x1,\y2-\y1)} in (\p1) ellipse (\n1 and \n2); 
}
		    

\begin{tikzpicture}[decoration=brace]

	\node (applogic) [roundBox, xshift = 52.5mm, minimum width = 20.5cm, text width = 5cm, fill = bgcolor1] at (0, 1.3cm) {Application};

	\node (cqrs) [roundBox, label={[anchor=west]}, fill = bgcolor0, minimum width = 10cm, minimum height = 8cm] at (0,-4.0cm) {};
	\node (retro) [roundBox, label={[anchor=west]}, fill = bgcolor2, minimum width = 10cm, minimum height = 8cm] at (10.5,-4.0cm) {};

	\node (api) [roundBox] {CQRS Application Interface};


	%\node (where) [align = center, yshift = 15mm, right = 10mm of applogic] {Where are the retroactive\\features available?};

	%\node [tri, shape border rotate=180] (A) {A};

	\node (commandmodel) [roundBox, below right = 10mm and 10mm of api] {Command Model};
	\draw [arrow, bend angle=25, bend left] 
		(api) to 
		node[midway,right=0.4em] {Commands}  
		(commandmodel);
	%\draw [arrow, bend angle=25, bend left, loosely dotted] 
		%(commandmodel) to 
		%node[midway,left=-3mm,yshift=-4mm,align=center] {Event\\Reference}  
		%(api);

	\repositoryA[5.2em,5em,2.0,eventlog, (3.4,-6.5)] { Event\\Log };
	\draw [arrow] 
		([yshift=0mm]commandmodel) to 
		node[midway,right=0.0em, text centered] {Events}  
		(eventlog);

	\node (querymodel) [roundBox,  below left = 10mm and 10mm of api, align = center] {Query\\Model};
	\draw [arrow, bend angle=25, bend right] 
		(api) to 
		node[midway,left=0.5em] {Queries}  
		(querymodel);

	\draw [arrow, bend angle=25, bend right] (querymodel) to (api);

	\repositoryB[5.2em,5em,2.0,currentstate, (-3.4,-6.5)] { Current State };

	%\draw [<->,arrowboth] 
		%([yshift=-3mm]querymodel) to 
		%node[midway,right=0.1em, text centered, align = center] {Read-only\\Access}  
		%(currentstate);

	\draw [arrow] 
		(currentstate) to
		node[midway,right=0.0em, text centered, text width = 1.5cm] {Read-only\\Access}  
		([yshift=-8.7mm]querymodel);

	%\draw [arrow, bend angle=25, bend left, dashed] 
		%(commandmodel) to 
		%node[midway,below=1.4em,text width=2.2cm, text centered] {State Updates (Events)}  
		%(currentstate);

	\draw [arrow, dashed] 
		(eventlog) to 
		node[midway,below=0.3em,text width=2.2cm, text centered] {State Updates (Events)}  
		(currentstate);


	% Retroactive Stuff
	\node (rapi) [roundBox, yshift = 40mm, right = 44mm of cqrs] {Retroactive Application Interface};

	\node (rcommandmodel) [roundBox, below right = 10mm and 10mm of rapi] {
		%\footnotesize
		Retroactive Operations Model
	};
	\draw [arrow, bend angle=25, bend left] 
		(rapi) to 
		node[midway, right=0.8em, align = center] {Retroactive\\Operations}  
		(rcommandmodel);
	\draw [arrow, bend angle=25, bend left] (rcommandmodel) to (rapi);

	\node (rquerymodel) [roundBox,  below left = 10mm and 10mm of rapi, align = center] {
		%\footnotesize
		Retroactive Query Model
	};
	%Retroactive Query Model + Application Query Model};
	\draw [arrow, bend angle=25, bend right] 
		(rapi) to 
		node[midway, left=0.4em, align = center] {Retroactive\\Queries}  
		(rquerymodel);

	\draw [arrow, bend angle=25, bend right] (rquerymodel) to (rapi);
	\repositoryA[5.2em,6em,2.0,rstore, (10.6,-6.5)] { Retroactive\\Store };

	\draw [<->,arrow] 
		%([yshift=-3mm]rquerymodel) to 
		(rstore) to 
		node[midway,left=0.7em, text centered, align = center] {Read-only\\Access}  
		(rquerymodel);

	\draw [<->,arrow] 
		([yshift=-3mm]rcommandmodel) to 
		node[midway,right=0.7em, align = center] {Write\\Access}  
		(rstore);

	\draw[decorate, yshift=-4ex] (5,-8) -- node[below=0.4ex] {eventually consistent} (-5,-8);

	\draw[decorate, yshift=-4ex] (15.5,-8) -- node[below=0.4ex] {always consistent} (5.5,-8);

	\draw [arrow,dotted] 
		(eventlog) to 
		node[midway,below=0.1em, text centered, align = center] {Read-only\\Access}  
		(rstore);

\end{tikzpicture}